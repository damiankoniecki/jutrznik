%\psalmusverse=0\relax
%\startlines\startnarrower[0mm]%
\startpsalmus[title={Niechaj będzie uwielbiony (por Łk 1,68-79)}]
Niechaj będzie uwielbiony Bóg nasz Pan nad pany\pmed
Bo nawiedził i wyzwolił lud Swój, lud wybrany.\pfin
Bo nas podniósł i moc Jego znowu wśród nas widać\pmed
Przez tych świętych, których znalazł z potomstwa Dawida.\pfin
Według Swojej obietnicy, którą w czasach dawnych\pmed
Ogłaszali nam prorocy słowem Jego sławni.\pfin
Przyrzekł przez nich, że wybawi nas od nieprzyjaciół,\pmed
I że żadnej nienawiści nie da nas wytracić.\pfin
Przyrzekł, że dla naszych ojców będzie miłosierny,\pmed
Że Przymierza nie zapomni, że nam będzie wierny.\pfin
Że dotrzyma swej przysięgi wobec Abrahama,\pmed
Który stał się ojcem wiernych, gdy się nie załamał.\pfin
Przyrzekł, że my dzieci wiary, wolni z ręki wrogów,\pmed
Niewzruszenie będziem mogli służyć tylko Bogu.\pfin
W pobożności wiodąc życie i sprawiedliwości\pmed
Przez dzień każdy, coraz bliżej spotkania w wieczności.\pfin
Tak Cię dzisiaj, dziecię swoje, Pan Bóg potrzebuje,\pmed
Byś szedł przed Nim jako prorok, co drogę gotuje.\pfin
I wśród ludu Jego głosił naukę zbawienia,\pmed
Aby wszyscy dostąpili grzechów przebaczenia.\pfin
Bo w Swym Sercu się lituje Pan Bóg Wszechmogący,\pmed
I dlatego wciąż przychodzi jak Wschodzące Słońce.\pfin
By rozproszyć grozę śmierci tym, co żyją w mroku,\pmed
I nawrócić ich na drogę, która daje pokój.\pfin
Chwała, cześć i uwielbienie w Trójcy Jedynemu.\pmed
Bogu Ojcu i Synowi, Duchowi Świętemu:\pfin
Temu, który był w wieczności, który jest na niebie,\pmed
który przyjdzie w końcu czasów, by nas wziąć do siebie.\pfin
%\stoppsalmus
\endinput

% vim: ft=context
