\usemodule[breviarium]
\enableregime[utf]
\mainlanguage[pl]
\enabletrackers[fonts.missing]
\setupinteraction[
    title={Jutrznia w Oazie Nowego Życia 2°},
    author={Diecezjalna Diakonia Liturgiczna Archidiecezji Warszawskiej},
    state=start,
]

\setupbackend[
    format={pdf/a-1a:2005},
    profile={default_cmyk.icc,default_rgb.icc,default_gray.icc},
    intent={ISO coated v2 300\letterpercent\space (ECI)},
]

\setupstructure[state=start,method=auto]
\setuppapersize[A5]
\setuplayout[location={middle,middle}]
\setupbodyfont[10.5pt]


\starttext

\startbooktitle[title={Jutrznia}]

\page[yes]

\startday[title={Niewola}] % {{{1

\starthour[title={Wezwanie}]

\ant Uwielbiajmy Pana i~Króla,~/ który do nas przyjdzie.

\startrubrica
Psalm Wezwania jak w~Częściach stałych, s.~FIXME.
\stoprubrica

\starthour[title={Jutrznia}]

\hourpartx{Hymn}{(piosenka dnia)}

\starthourpart[title={Psalmodia}]

\ant[n=1] Kiedy przyjdę~* i~ujrzę oblicze Boże?

\startpsalmus[title={Psalm 42}]
Jak łania pragnie wody ze strumieni,\pmed
tak dusza moja pragnie Ciebie, Boże.\pfin
Dusza moja Boga pragnie, Boga żywego,\pmed
kiedyż więc przyjdę i~ujrzę oblicze Boże?\pfin
Łzy są moim chlebem we dnie i~w~nocy;\pmed
\startquote Gdzie jest twój Bóg?\stopquote\space pytają mnie co dzień.\pfin
Rozpływa się we mnie moja dusza,\pmed
gdy wspominam, jak z~tłumem kroczyłem do Bożego domu\pfin
W~świątecznym orszaku,\pmed
wśród głosów radości i~chwały.\pfin
Czemu zgnębiona jesteś, duszo moja,\pmed
i~czemu trwożysz się we mnie?\pfin
Ufaj Bogu, bo jeszcze wysławiać Go będę:\pmed
On zbawieniem mojego oblicza i~moim Bogiem!\pfin
A~we mnie samym dusza przygnębiona,\pflx
przeto wspominam Cię z~ziemi Jordanu,\pmed
z~ziemi Hermonu i~góry Misar.\pfin
Głębia przyzywa głębię hukiem wodospadów.\pmed
Wszystkie Twe nurty i~fale nade mną się przewalają.\pfin
Niech Pan udzieli mi we dnie swej łaski,\pflx
a~w~nocy będę Mu śpiewał,\pmed
będę sławił Boga mego życia.\pfin
Mówię do Boga: Opoko moja, czemu zapominasz o~mnie?\pmed
Czemu chodzę smutny, gnębiony przez wroga?\pfin
Kości we mnie się kruszą,\pmed
gdy lżą mnie przeciwnicy,\pfin
Gdy cały dzień mówią do mnie:\pmed
\startquote Gdzie jest Bóg twój?\stopquote\pfin
Czemu zgnębiona jesteś, duszo moja,\pmed
i~czemu trwożysz się we mnie?\pfin
Ufaj Bogu, bo jeszcze wysławiać Go będę:\pmed
On zbawieniem mojego oblicza i~moim Bogiem!\pfin
\stoppsalmus

\antr

\ant[n=2] Okaż nam, Panie,~* miłosierdzie swoje.

\startpsalmus[title={Pieśń (Syr 36,1-5. 10-13)}]
Zmiłuj się nad nami, Panie, Boże wszystkich rzeczy,\pmed
i~spójrz i~ześlij bojaźń przed Tobą na wszystkie narody.\pfin
Wyciągnij rękę przeciw obcym ludom,\pmed
aby ujrzały Twoją potęgę.\pfin
Bo jak przez nas okazałeś im świętość swoją,\pmed
tak przez nich wobec nas okaż się wielkim.\pfin
Niech i~one uznają, jak my uznajemy,\pmed
że nie ma Boga, oprócz Ciebie, Panie.\pfin
Powtórz znaki i~znów uczyń cuda,\pmed
wsław swoją rękę i~ramię prawe.\pfin
Zgromadź wszystkie pokolenia Jakuba\pmed
i~weź je w~posiadanie, jak było na początku.\pfin
Panie, zlituj się nad narodem nazwanym Twoim imieniem,\pmed
nad Izraelem, którego uznałeś za pierworodnego.\pfin
Miej miłosierdzie nad Twym świętym miastem,\pmed
nad Jeruzalem, miejscem Twego odpoczynku.\pfin
Napełnij Syjon wysławianiem Twej mocy,\pmed
a~Twój lud swoją chwałą.\pfin
\stoppsalmus

\antr

\ant[n=3] Błogosławiony jesteś, Panie,~* na sklepieniu nieba.

\startpsalmus[title={Psalm 19 A, 2-7}]
Niebiosa głoszą chwałę Boga,\pmed
dzieło rąk Jego obwieszcza nieboskłon.\pfin
Dzień opowiada dniowi,\pmed
noc nocy wiadomość przekazuje.\pfin
Nie są to słowa ani nie jest to mowa,\pmed
których by dźwięku nie usłyszano.\pfin
Ich głos się rozchodzi po całej ziemi,\pmed
ich słowa aż po krańce świata.\pfin
Tam słońcu namiot postawił,\pflx
a~ono jak oblubieniec wychodzi ze swej komnaty,\pmed
cieszy się jak siłacz ruszający do biegu.\pfin
Ono wschodzi na krańcu nieba\pflx
i~biegnie aż po drugi kraniec,\pmed
a~nic przed jego żarem się nie schroni.\pfin
\stoppsalmus

\antr

\starthourpart[title={Czytanie\hfill\tf Dz 7,17-29a}]
W~miarę, jak zbliżał się czas obietnicy, którą Bóg dał Abrahamowi, rozrastał
się lud i~rozmnażał w~Egipcie, aż doszedł do władzy inny król w~Egipcie, który
nic nie wiedział o~Józefie. Działał on podstępnie przeciwko naszemu narodowi
i~przymuszał ojców naszych do wyrzucania niemowląt, aby nie zostawały przy
życiu. Wówczas właśnie narodził się Mojżesz. Był on miły Bogu. Przez trzy
miesiące karmiono go w~domu ojca. A~gdy go wyrzucono, zabrała go córka faraona
i~przybrała go sobie za syna. Mojżesza wykształcono we wszystkich naukach
egipskich, i~potężny był w~słowie i~czynie. Gdy skończył lat czterdzieści,
przyszło mu na myśl odwiedzić swych braci, synów Izraela. I~zobaczył jednego,
któremu wyrządzono krzywdę. Stanął w~jego obronie i~zabiwszy Egipcjanina
pomścił skrzywdzonego. Sądził, że bracia jego zrozumieją iż Bóg przez jego ręce
daje im wybawienie, lecz oni nie zrozumieli. Następnego dnia zjawił się wśród
nich, kiedy bili się między sobą, i~usiłował ich pogodzić. Ludzie, braćmi
jesteście ---~zawołał ---~czemuż krzywdzicie jeden drugiego? Ten jednak, który
krzywdził bliźniego, odepchnął go. Któż ciebie ustanowił panem i~sędzią nad
nami? ---~zawołał ---~czy chcesz mnie zabić, tak jak wczoraj zabiłeś
Egipcjanina? Na te słowa Mojżesz uciekł i~żył jako cudzoziemiec w~ziemi Madian.

\starthourpart[title={Pieśń Zachariasza}]

\ant[title={Ant. do pieśni Zachariasza}] Bóg przez wielką swą miłość, jaką nas
ukochał,~* zesłał swego Syna w~ciele podobnym do ciała grzesznego.

\startrubrica
Pieśń Zachariasza w~Częściach stałych, s.~FIXME.
\stoprubrica

\starthourpart[title={Prośby}]

Obchodząc uroczystość Zwiastowania, czcimy początek dzieła naszego
zbawienia. Z~radością więc skierujmy do Boga wspólne błagania:

\aklamacja Niech Bogarodzica wstawia się za nami.

\prosba Spraw, Boże, abyśmy z radością przyjęli naszego Zbawiciela,
--- jak Maryja Dziewica ochoczo przyjęła nowinę zwiastowaną Jej przez anioła.

\prosba O~dobry Ojcze, pamiętaj o~nas i~o~wszystkich ludziach,
--- jak wejrzałeś na pokorę swojej służebnicy.

\prosba Spraw, abyśmy zawsze zgadzali się z~Twoją wolą,
--- jak Maryja, nowa Ewa, posłusznie przyjęła Twoje Boskie Słowo.

\prosba Niechaj Maryja wspomaga ubogich, podtrzymuje małodusznych, pociesza
płaczących,
--- niech modli się za lud, oręduje za duchowieństwem, wstawia się za
poświęconymi Bogu kobietami.

Ojcze nasz.

\starthourpart[title={Modlitwa}]
\begingroup\catcode`*=\active
Wszechmogący Boże, od dawna przygniata nas jarzmo grzechów,~* spraw, aby nas
wyzwoliło upragnione nowe narodzenie Twojego Syna. Który z~Tobą żyje i~króluje
w~jedności Ducha Świętego,~* Bóg, przez wszystkie wieki wieków.
\endgroup


\startday[title={Bóg Wybawiciel}] % {{{1

Wezwanie: Przyjdźcie, uwielbiajmy Pana, * który jest Pasterzem
swojego ludu
Hymn: piosenka dnia
Psalmodia
1 ant. Boże, Twoja droga jest święta, * nikt nie dorówna wielkością
naszemu Bogu.

\startpsalmus[title={Psalm 77}]
Głos mój się wznosi do Boga, gdy wołam,\pmed
głos mój wznoszę do Boga, aby mnie usłyszał.\pfin
W~dniu mej niedoli szukam Pana,\pmed
w~nocy niestrudzenie wyciągam rękę.\pfin
Dusza moja jest niepocieszona,\pflx
jęczę, kiedy wspomnę Boga,\pmed
słabnie mój duch, gdy rozmyślam.\pfin
Ty spędzasz sen z~moich powiek,\pmed
z~niepokoju mówić nie potrafię.\pfin
Rozpamiętuję dni, które dawno minęły,\pmed
i~lata poprzednie wspominam.\pfin
Rozmyślam nocą w~swym sercu,\pmed
rozważam, a~duch mój docieka:\pfin
\startquote Czy Bóg odrzuci na wieki\pmed
i~już nie będzie łaskawy?\pfin
Czy Jego łaskawość ustała na zawsze\pmed
i~słowo zamilkło na pokolenia?\pfin
Czy Bóg zapomniał o~litości,\pmed
czy w~gniewie powstrzymał swe miłosierdzie?\stopquote\pfin
I~mówię: \startquote Jakże to bolesne,\pmed
że odwróciła się ode mnie prawica Najwyższego\stopquote.\pfin
Wspominam dzieła Pana,\pmed
oto wspominam Twoje dawne cuda.\pfin
Rozmyślam o~wszystkich Twych dziełach\pmed
i~czyny Twoje wspominam.\pfin
Boże, Twoja droga jest święta,\pmed
który z~bogów dorówna wielkością naszemu Bogu?\pfin
Ty jesteś Bogiem działającym cuda,\pmed
ludziom objawiłeś swą potęgę.\pfin
Ramieniem swoim Twój lud wybawiłeś,\pmed
synów Jakuba i~Józefa.\pfin
Boże, ujrzały Cię wody,\pflx
ujrzały Cię wody i~zadrżały,\pmed
wzburzyły się ich odmęty.\pfin
Chmury wylały wody,\pflx
zahuczały chmury\pmed
i~Twoje strzały się posypały.\pfin
Głos Twego grzmotu jak łoskot wozu,\pflx
pioruny świat rozjaśniły,\pmed
ziemia poruszyła się i~zatrzęsła.\pfin
Twoja droga wiodła przez wody,\pflx
Twoja ścieżka przez wodne obszary\pmed
i~nie znać było Twych śladów.\pfin
Wiodłeś Twój lud jak trzodę\pmed
ręką Mojżesza i~Aarona.\pfin
\stoppsalmus

Ant. Boże, Twoja droga jest święta, / nikt nie dorówna wielkością
naszemu Bogu.
2 ant. Moje serce raduje się w Panu, * który poniża i wywyższa.
Pieśń (1 Sm 2, 1-10)
Moje serce raduje się w Panu, *
dzięki Niemu moc moja wzrasta.
Szeroko otwarłam usta przeciw moim wrogom, *
bo cieszyć się mogę Twoją pomocą.
Nikt nie jest tak święty jak Ty, Panie, q
poza Tobą bowiem nie ma nikogo, *
prócz naszego Boga nie ma innej ostoi.
Nie powtarzajcie słów pełnych pychy, *
niech mowa harda z ust waszych nie wychodzi,
Gdyż Pan jest Bogiem wszechwiedzącym *
i On ocenia uczynki.
Łuk potężnych się łamie, *
a mocą przepasują się słabi.
Syci za chleb się najmują, *
głodni zaś odpoczywają.
Niepłodna rodzi siedmioro, *
a matka wielu dzieci usycha.
Pan daje śmierć i życie, *
wtrąca do Otchłani i z niej wyprowadza.
Pan czyni ubogim lub bogatym, *
poniża i wywyższa.
Biedaka z prochu podnosi, *
z błota dźwiga nędzarza,
By go wśród książąt posadzić *
i dać mu tron chwały.
Fundamenty ziemi należą do Pana *
i na nich świat On położył.
On strzeże kroków swoich wiernych, q
grzesznicy zaś zginą w ciemnościach, *
bo nie własną siłą człowiek zwycięża.
Pan wniwecz opornych obraca *
i przeciw nim grzmi na niebiosach.
Pan sądzi krańce ziemi, q
króla obdarza potęgą *
i wywyższa moc swego pomazańca.
Chwała Ojcu i Synowi, *
i Duchowi Świętemu.
Jak była na początku, teraz i zawsze, *
i na wieki wieków. Amen.
Ant. Moje serce raduje się w Panu, / który poniża i wywyższa.
3 ant. Pan króluje, * wesel się, ziemio. q
Psalm 97
Pan króluje, wesel się, ziemio, *
q radujcie się, liczne wyspy!
Obłok i ciemność wokół Niego, *
prawo i sprawiedliwość podstawą Jego tronu.



Przed Jego obliczem idzie ogień *
i dokoła pożera nieprzyjaciół Jego.
Jego błyskawice wszechświat rozświetlają, *
a ziemia drży na ten widok.
Góry jak wosk topnieją przed obliczem Pana, *
przed obliczem Władcy całej ziemi.
Jego sprawiedliwość rozgłaszają niebiosa *
i wszystkie ludy widzą Jego chwałę.
Niech zawstydzą się wszyscy, którzy czczą posągi q
i chlubią się bożkami. *
Niech wszystkie bóstwa hołd Mu oddają!
Słyszy o tym i cieszy się Syjon, q
radują się miasta Judy *
z Twoich wyroków, o Panie.
Ponad całą ziemią Tyś bowiem wywyższony *
i nieskończenie wyższy od wszystkich bogów.
Pan tych miłuje, którzy zła nienawidzą, q
On strzeże dusz świętych swoich, *
wydziera je z rąk grzeszników.
Światło wschodzi dla sprawiedliwego *
i radość dla ludzi prawego serca.
Weselcie się w Panu, sprawiedliwi, *
i sławcie Jego święte imię.
Chwała Ojcu i Synowi, *
i Duchowi Świętemu.
Jak była na początku, teraz i zawsze, *
i na wieki wieków. Amen.
Ant. Pan króluje, wesel się, ziemio.
Czytanie: Wj 6,2-6
Bóg rozmawiał z Mojżeszem i powiedział mu: Jam jest Jahwe. Ja objawiłem się Abrahamowi, Izaakowi i Jakubowi jako Bóg Wszechmocny, ale
imienia mego, Jahwe, nie objawiłem im. Ponadto ustanowiłem też przymierze moje z nimi, że im dam kraj Kanaan, kraj ich wędrówek, gdzie przebywali jako przybysze. Ja także usłyszałem jęk Izraelitów, których Egipcjanie
obciążyli robotami, i wspomniałem na moje przymierze. Przeto powiedz
synom izraelskim: Ja jestem Pan! Uwolnię was od jarzma egipskiego i wybawię was z niewoli, i wyswobodzę was wyciągniętym ramieniem i przez
surowe kary.



Ant do pieśni Zachariasza:
Gdy Elżbieta usłyszała pozdrowienie Maryi, * wydała okrzyk i powiedziała: /
A skądże mi to, że Matka mojego Pana przychodzi do mnie.
Prośby
Oddając cześć naszemu Zbawicielowi, który narodził się z Maryi Dziewicy,
zanośmy do Niego pokorne błagania:
Niech Twoja Matka wstawia się za nami.
Jezu, Słońce sprawiedliwości, Twoje przyjście poprzedziła Niepokalana
Dziewica, jak pełna blasku jutrzenka,
– spraw, abyśmy zawsze żyli w promieniach Twojej światłości.
Dozwól nam, Panie, naśladować Twoją Matkę, która obrała najlepszą cząstkę,
– spraw, abyśmy za Jej przykładem szukali pokarmu dającego życie wieczne.
Zbawicielu świata, Ty mocą swojego odkupienia zachowałeś Twoją Matkę
od wszelkiej zmazy grzechu,
– zachowaj nas od skażenia grzechem.
Nasz Odkupicielu, Ty sprawiłeś, że Maryja Dziewica stała się godnym Ciebie mieszkaniem i przybytkiem Ducha Świętego,
– daj, abyśmy byli na wieki świątynią Twego Ducha.
Ojcze nasz.
Modlitwa:
Okaż swą potęgę, Panie, i przybądź, q niech Twoja opieka wyzwoli nas od
i niebezpieczeństw grożących nam wskutek naszych grzechów, * a Twoja
moc niech nas zbawi. Który żyjesz i królujesz z Bogiem Ojcem w jedności
Ducha Świętego, * Bóg, przez wszystkie wieki wieków.



\startday[title={Bóg z~nami}] % {{{1
Wezwanie: Chrystus nam się narodził, * uwielbiajmy Wcielone Słowo
Hymn: piosenka dnia
Psalmodia
1 ant. Rano głosimy Twą łaskawość, Panie, * a wierność Twoją nocami.
Psalm 92
Dobrze jest dziękować Panu, *
śpiewać Twojemu imieniu, Najwyższy,
Rano głosić łaskawość Twoją, *
a wierność Twoją nocami,
Na harfie dziesięciostrunnej i lirze, *
pieśnią przy dźwiękach cytry.
Bo weselę się, Panie, Twoimi czynami, *
raduję się dziełami rąk Twoich.
Jak wielkie są dzieła Twoje, Panie, *
i jakże głębokie Twe myśli!
Nie zna ich człowiek nierozumny *
i głupiec ich nie pojmuje.
Chociaż występni się pienią jak zielsko, *
a złoczyńcy jaśnieją przepychem,
I tak pójdą na wieczną zagładę, *
Ty zaś, Panie, na wieki jesteś wywyższony.
Bo oto wrogowie Twoi, Panie, q
bo oto wrogowie Twoi poginą, *
rozproszą się wszyscy złoczyńcy.
Dałeś mi siłę bawołu, *
skropiłeś mnie świeżym olejkiem.
Moje oko spogląda z góry na nieprzyjaciół, q
a uszy me usłyszały o klęsce przeciwników, *
tych, którzy na mnie powstają.
Sprawiedliwy zakwitnie jak palma, *
rozrośnie się jak cedr na Libanie.
Zasadzeni w domu Pańskim *
rozkwitną na dziedzińcach Boga naszego.


Nawet i w starości wydadzą owoc, *
zawsze pełni życiodajnych soków,
Aby świadczyć, że Pan jest sprawiedliwy; *
On moją Opoką i nie ma w Nim nieprawości.
Chwała Ojcu i Synowi *
i Duchowi Świętemu.
Jak była na początku, teraz i zawsze, *
i na wieki wieków. Amen.
Ant. Rano głosimy Twą łaskawość, Panie, / a wierność Twoją nocami.
2 ant. Uznajcie wielkość * naszego Boga.
Pieśń (Pwt 32, 1-12)
Uważajcie, niebiosa, na to, co powiem, *
słuchaj, ziemio, głosu mojego.
Jak deszcz niech spływa moje pouczenie, *
jak rosa niech pada me słowo,
Jak deszcz rzęsisty na zieleń, *
jak deszcz życiodajny na trawę!
Oto będę głosić imię Pana; *
uznajcie wielkość naszego Boga!
On jest Opoką, a Jego dzieło doskonałe, *
wszystkie Jego drogi są słuszne.
On jest Bogiem wiernym i nie zawodzi, *
On sprawiedliwy i prawy.
Przestali być Jego dziećmi, q
bo grzech popełnili, *
pokolenie przewrotne i zakłamane.
Więc tak chcesz odpłacić Panu, *
ludu głupi i bezrozumny?
Czyż nie On jest twoim Ojcem i Stwórcą, *
który cię uczynił i dał ci życie?
Wspomnij na dni, które przeminęły, *
rozważ lata poprzednich pokoleń.
Zapytaj swego ojca, by cię pouczył, *
i twoich starców, niech ci opowiedzą,
Jak to Najwyższy obdarzał dziedzictwem narody *
i rozdzielał synów człowieczych.



Wtedy wytyczył granice dla ludów *
według liczby synów Boga sprawiedliwego,
Bo Jego lud jest własnością Pana, *
Jakub Jego wyłącznym dziedzictwem.
Znalazł go na ziemi pustynnej, *
na odludziu, gdzie brzmiały tylko dzikie głosy.
Opiekował się nim i pouczał, *
strzegł jak źrenicy oka.
Jak orzeł, który krąży nad gniazdem, *
by z niego wywabić swe pisklęta,
I bierze je na skrzydła rozpostarte, *
niosąc je na samym sobie,
Tak Pan go prowadził, *
a nie było z nim obcego boga.
Chwała Ojcu i Synowi, *
i Duchowi Świętemu.
Jak była na początku, teraz i zawsze, *
i na wieki wieków. Amen.
Ant. Uznajcie wielkość naszego Boga.
3 ant. Panie, jak przedziwne jest Twoje imię * na całej ziemi.
Psalm 8
O Panie, nasz Panie, q
jak przedziwne jest Twoje imię na całej ziemi! *
Tyś swój majestat wyniósł nad niebiosa.
Sprawiłeś, że na przekór Twoim przeciwnikom q
usta dzieci i niemowląt oddają Ci chwałę, *
aby poskromić nieprzyjaciela i wroga.
Gdy patrzę na Twe niebo, dzieło palców Twoich, *
na księżyc i gwiazdy, któreś Ty utwierdził:
Czym jest człowiek, że o nim pamiętasz, *
czym syn człowieczy, że troszczysz się o niego?
Uczyniłeś go niewiele mniejszym od aniołów, *
uwieńczyłeś go czcią i chwałą.
Obdarzyłeś go władzą nad dziełami rąk Twoich, *
wszystko złożyłeś pod jego stopy:
Owce i bydło wszelakie, *
i dzikie zwierzęta,



Ptaki niebieskie i ryby morskie, *
wszystko, co szlaki mórz przemierza.
O Panie, nasz Panie, *
jak przedziwne jest Twoje imię na całej ziemi!
Chwała Ojcu i Synowi, *
i Duchowi Świętemu.
Jak była na początku, teraz i zawsze, *
i na wieki wieków. Amen.
Ant. Panie, jak przedziwne jest Twoje imię / na całej ziemi.
Czytanie: Mt 1,20-23
Gdy Józef powziął tę myśl, oto anioł Pański ukazał mu się we śnie i rzekł:
Józefie, synu Dawida, nie bój się wziąć do siebie Maryi, twej Małżonki; albowiem z Ducha Świętego jest to, co się w Niej poczęło. Porodzi Syna,
któremu nadasz imię Jezus, On bowiem zbawi swój lud od jego grzechów.
A stało się to wszystko, aby się wypełniło słowo Pańskie powiedziane przez
Proroka: Oto Dziewica pocznie i porodzi Syna, któremu nadadzą imię Emmanuel, to znaczy: Bóg z nami.
Ant do pieśni Zachariasza:
Chwała na wysokościach Bogu, * a na ziemi pokój ludziom, których umiłował,
/ alleluja.
Prośby
Uwielbiajmy Słowo Boże, które istniejąc od wieków, w pełni czasów stało
się ciałem i zamieszkało między nami. Z radością więc wołajmy do Niego:
Głośmy z weselem: Bóg jest między nami.
Chryste, Słowo Przedwieczne, Ty przychodząc na ziemię opromieniasz ją
radością,
– napełnij nasze serca łaską Twego nawiedzenia.
Nasz Zbawicielu, Ty przez swe narodzenie objawiasz nam wierność Boga,
– spraw, abyśmy wiernie dochowali przyrzeczeń chrztu świętego.
Królu nieba i ziemi, Ty posłałeś aniołów, aby ludziom głosili pokój,
– zachowaj nasze życie w Twoim pokoju.
Panie, Ty przyszedłeś jako prawdziwy krzew winny dający nam życie,
– spraw, abyśmy jak gałązki trwali zawsze w Tobie i przynosili obfite owoce.



Ojcze nasz.
Modlitwa:
Boże, Ty w przedziwny sposób stworzyłeś człowieka i w jeszcze cudowniejszy sposób odnowiłeś jego godność, q daj nam uczestniczyć w bóstwie
Twojego Syna, * który przyjął naszą ludzką naturę. Który z Tobą żyje i króluje
w jedności Ducha Świętego, * Bóg, przez wszystkie wieki wieków.



\startday[title={Bóg wzywa}] % {{{1

Wezwanie: Chrystus nam się objawił, * pokłon Jemu oddajmy
Hymn: piosenka dnia
Psalmodia:
1 ant. Gotowe jest serce moje, Boże, * gotowe jest serce moje,
/ zaśpiewam psalm i zagram. q

\startpsalmus[title={Psalm 108}]
Gotowe jest serce moje, Boże,\pflx
gotowe jest serce moje,\pmed
zaśpiewam psalm i~zagram,\pfin
q~Zbudź się, duszo moja,\pflx
zbudź się, harfo i~cytro,\pmed
a~ja obudzę jutrzenkę.\pfin
Będę Cię chwalił wśród ludów, Panie,\pmed
zaśpiewam Ci psalm wśród narodów.\pfin
Bo Twoja łaska sięga aż do nieba,\pmed
a~wierność Twoja po chmury.\pfin
Wznieś się ponad niebiosa, Panie,\pmed
nad całą ziemię Twoja chwała.\pfin
Aby ocaleli, których Ty miłujesz,\pmed
wspomóż nas Twoją prawicą i~wysłuchaj.\pfin
Bóg przemówił w~swojej świątyni:\pflx
\startquote Będę się radował i~podzielę Sychem,\pmed
a~dolinę Sukkot wymierzę.\pfin
Do mnie należy Gilead i~ziemia Manassesa,\pflx
Efraim jest szyszakiem mej głowy,\pmed
Juda berłem moim.\pfin
Moab jest moją misą do mycia,\pflx
na Edomie mój but postawię,\pmed
zatriumfuję nad Filisteą!\stopquote\pfin
Któż mnie wprowadzi do miasta warownego?\pmed
Któż mnie doprowadzi do Edomu?\pfin
Czyż nie Ty, Boże, który nas odrzuciłeś\pmed
i~już nie wychodzisz, Boże, z~naszymi wojskami?\pfin
Daj nam pomoc przeciw nieprzyjacielowi,\pmed
bo ludzkie wsparcie jest zawodne.\pfin
Dokonamy w~Bogu czynów pełnych mocy,\pmed
a~On podepcze naszych nieprzyjaciół.\pfin
\stoppsalmus

Ant. Gotowe jest serce moje, Boże, / gotowe jest serce moje,
/ zaśpiewam psalm i zagram.
2 ant. Dam wam serce nowe * i nowego ducha tchnę w wasze wnętrze.

\startpsalmus[title={Pieśń (Ez 36, 24-28)}]
Wezmę was spośród ludów,\pflx
zgromadzę was ze wszystkich krajów\pmed
i~przywiodę z~powrotem do waszej ziemi.\pfin
Pokropię was czystą wodą,\pmed
i~staniecie się czyści.\pfin
Obmyję was z~wszelkiej nieczystości\pmed
i~z~waszego bałwochwalstwa.\pfin
Dam wam serce nowe\pmed
i~nowego ducha tchnę w~wasze wnętrze.\pfin
Wyjmę z~was serce kamienne,\pmed
i~dam wam serce z~ciała.\pfin
Tchnę w~was mojego Ducha\pmed
i~sprawię, że będziecie żyć według mych nakazów,\pfin
Że będziecie przestrzegać przykazań\pmed
i~postępować zgodnie z~nimi.\pfin
Wtedy zamieszkacie w~kraju,\pmed
który dałem waszym przodkom,\pfin
I~będziecie moim ludem,\pmed
Ja zaś będę Bogiem waszym.\pfin
Chwała Ojcu i~Synowi,\pmed
i~Duchowi Świętemu.\pfin
Jak była na początku, teraz i~zawsze,\pmed
i~na wieki wieków. Amen.\pfin
\stoppsalmus



Ant. Dam wam serce nowe / i nowego ducha tchnę w wasze wnętrze.
3 ant. Będę chwalił Pana * do końca mego życia.
Psalm 146
Chwal, duszo moja, Pana; q
będę chwalił Pana do końca mego życia, *
będę śpiewał mojemu Bogu, dopóki istnieję.
Nie pokładajcie ufności w książętach *
ani w człowieku, który zbawić nie może.
Kiedy duch go opuści, znów w proch się obraca *
i przepadają wszystkie jego zamiary.
Szczęśliwy ten, kogo wspiera Bóg Jakuba, *
kto pokłada nadzieję w Panu Bogu.
On stworzył niebo i ziemię, i morze *
ze wszystkim, co w nich istnieje.
On wiary dochowuje na wieki, *
uciśnionym sprawiedliwość wymierza,
Chlebem karmi głodnych, *
wypuszcza na wolność uwięzionych.
Pan przywraca wzrok ociemniałym, q
Pan dźwiga poniżonych, *
Pan kocha sprawiedliwych.
Pan strzeże przybyszów, q
ochrania sierotę i wdowę, *
lecz występnych kieruje na bezdroża.
Pan króluje na wieki, *
Bóg twój, Syjonie, przez pokolenia.
Chwała Ojcu i Synowi, *
i Duchowi Świętemu.
Jak była na początku, teraz i zawsze,
i na wieki wieków. Amen.
Ant. Będę chwalił Pana do końca mego życia.
Czytanie: Wj 3,4
Gdy zaś Pan ujrzał, że Mojżesz podchodził, żeby się przyjrzeć, zawołał Bóg
do niego ze środka krzewu: Mojżeszu, Mojżeszu! On zaś odpowiedział: Oto
jestem.



Antyfona do pieśni Zachariasza:
Gdy rodzice wnosili do świątyni Dzieciątko Jezus, * Symeon wziął Je w objęcia i błogosławił Boga.
Prośby
Uwielbiajmy naszego Zbawiciela, który został ofiarowany Bogu w świątyni,
i zanośmy do Niego nasze prośby:
Niech nasze oczy ujrzą Twe zbawienie.
Jezu Chryste, Ty zechciałeś, aby zgodnie z Prawem ofiarowano Cię w świątyni,
– naucz nas samych siebie składać w ofierze Kościoła razem z Tobą.
Jezu, Pociecho Izraela, sprawiedliwy Symeon przyszedł do świątyni na Twoje
spotkanie,
– spraw, abyśmy spotykali Ciebie w naszych braciach.
Jezu, Nadziejo narodów, o Tobie prorokini Anna mówiła wszystkim, którzy
oczekiwali odkupienia Izraela,
– naucz nas godnie o Tobie mówić do wszystkich ludzi.
Jezu, kamieniu węgielny królestwa Bożego, Ty zostałeś postawiony jako znak
sprzeciwu,
– spraw, aby wszyscy ludzie przez wiarę i miłość osiągnęli z Tobą chwałę
zmartwychwstania.
Ojcze nasz.
Modlitwa:
Boże, Ty kierujesz swoim ludem przez pasterzy, q ześlij na swój Kościół ducha pobożności i męstwa * i powołaj ludzi, którzy będą godnie pełnili służbę
przy ołtarzu oraz mężnie i pokornie głosili Ewangelię. Przez naszego Pana
Jezusa Chrystusa, Twojego Syna, q który z Tobą żyje i króluje w jedności
Ducha Świętego, * Bóg, przez wszystkie wieki wieków.



\startday[title={Bóg posyła}] % {{{1
Wezwanie: Uwielbiajmy Chrystusa, umiłowanego Syna Bożego, w którym
Ojciec upodobał sobie
Hymn: piosenka dnia
Psalmodia
1 ant. Tobie chcę śpiewać, o Panie, * będę szedł drogą nieskalaną.

Psalm 101
Będę śpiewał o sprawiedliwości i łasce, *
Tobie chcę śpiewać, o Panie.
Będę szedł drogą nieskalaną; *
kiedyż Ty do mnie przyjdziesz?
Będę chodził z sercem niewinnym *
wewnątrz swojego domu.
Nie będę zwracał oczu *
ku sprawie niegodziwej.
W nienawiści mam czyny przestępcy, *
nie przylgną one do mnie.
Przewrotne serce będzie ode mnie z daleka, *
tego, co złe, nawet znać nie chcę.
Zniszczę każdego, kto skrycie uwłacza bliźniemu, *
pysznych oczu i serca nadętego nie zniosę.
Oczy swe zwracam na wiernych tej ziemi, *
aby ze mną zamieszkać mogli.
Ten, który kroczy drogą nieskalaną, *
będzie mi usługiwał.
Kto knuje podstęp, nie zamieszka w mym domu, *
nie ostoi się wobec mego wzroku, kto rozgłasza kłamstwa.
Każdego dnia będę tępił wszystkich grzeszników ziemi, *
wypędzę z miasta Pańskiego wszelkich złoczyńców.
Chwała Ojcu i Synowi, *
i Duchowi Świętemu.
Jak była na początku, teraz i zawsze, *
i na wieki wieków. Amen.


Ant. Tobie chcę śpiewać, o Panie, / będę szedł drogą nieskalaną.
2 ant. Nie odbieraj nam, Panie, * swego miłosierdzia.
Pieśń (Dn 3,26-27. 29. 34-41)
Błogosławiony jesteś, Panie, Boże naszych ojców, q
i pełen chwały, *
a Twoje imię jest błogosławione na wieki.
Jesteś sprawiedliwy we wszystkim, coś uczynił z nami, *
wszelkie Twoje dzieła są pełne prawdy.
Twoje drogi są proste, *
a wyroki najsłuszniejsze.
Myśmy zgrzeszyli i popełnili nieprawość *
odstępując od Ciebie.
We wszystkim przewrotność okazaliśmy *
i nie słuchaliśmy Twoich przykazań.
Nie opuszczaj nas na zawsze q
i ze względu na świętość Twego imienia *
nie zrywaj Twojego przymierza.
Nie odbieraj nam swego miłosierdzia *
przez wzgląd na Abrahama, przyjaciela Twego,
Twojego sługę, Izaaka, *
i Twego świętego, Izraela.
Im to przyrzekłeś, *
że rozmnożysz ich potomstwo
Jak gwiazdy na niebie *
i jak piasek na morskim brzegu.
Panie, oto jesteśmy najmniejsi *
spośród wszystkich narodów.
Oto dziś jesteśmy poniżeni na całej ziemi *
z powodu naszych grzechów.
Nie ma już władcy, proroka ani wodza, q
ani całopalenia, ani ofiar, *
ani daru pokarmów, ani kadzenia.
Nie ma gdzie złożyć Tobie daru z pierwszych płodów *
i doświadczyć miłosierdzia Twego.
Niech jednak dusza strapiona i duch uniżony *
znajdą upodobanie u Ciebie.
Tak jak całopalenia z baranów i cielców, *
i tysięcy tłustych owiec,



Niech się dziś stanie nasza ofiara dla Ciebie *
i spodoba się Tobie.
Bo ci, którzy ufność pokładają w Tobie, *
nie zaznają wstydu.
Teraz zaś z całego serca idziemy za Tobą, q
odczuwamy lęk wobec Ciebie *
i szukamy Twojego oblicza.
Chwała Ojcu i Synowi, *
i Duchowi Świętemu.
Jak była na początku, teraz i zawsze, *
i na wieki wieków. Amen.
Ant. Nie odbieraj nam, Panie, swego miłosierdzia.
3 ant. Boże, będę Ci śpiewał * pieśń nową.
Psalm 144,1-10
Błogosławiony Pan, Opoka moja, q
On moje ręce zaprawia do walki, *
moje palce do bitwy.
On mocą i warownią moją, *
osłoną moją i moim wybawcą,
Moją tarczą i schronieniem, *
On, który mi poddaje ludy.
Panie, czym jest człowiek, że troszczysz się o niego, *
czym syn człowieczy, że Ty o nim myślisz?
Do tchnienia wiatru podobny jest człowiek, *
dni jego jak cień przemijają.
Nachyl swych niebios i zstąp po nich, Panie, *
gór dotknij, aby zadymiły.
Rozprosz ich uderzeniem pioruna, *
wypuść swe strzały, by ich porazić.
Wyciągnij swą rękę z wysoka, *
wybaw mnie z wód ogromnych.
Uwolnij z rąk cudzoziemców, q
tych, których usta mówią kłamliwie, *
a prawica wznosi się do fałszywej przysięgi.
Boże, będę Ci śpiewał pieśń nową, *
grać Ci będę na harfie o dziesięciu strunach.



Ty królom dajesz zwycięstwo, *
Tyś wyzwolił Dawida, swego sługę.
Chwała Ojcu i Synowi, *
i Duchowi Świętemu.
Jak była na początku, teraz i zawsze, *
i na wieki wieków. Amen.
Ant. Boże, będę Ci śpiewał pieśń nową.
Czytanie: Iz 6,8-9a
I usłyszałem głos Pana mówiącego: Kogo mam posłać? Kto by Nam poszedł? Odpowiedziałem: Oto ja, poślij mnie! I rzekł mi: Idź i mów do tego
ludu.
Ant do pieśni Zachariasza:
Chrystus chrztem swoim uświęcił świat cały, * udzielił nam odpuszczenia
grzechów / i przez wodę i Ducha oczyścił nas wszystkich.
Prośby
Błagajmy naszego Odkupiciela, który zechciał przyjąć chrzest z rąk Jana
w Jordanie, i wołajmy do Niego:
Panie, zmiłuj się nad nami!
Chryste, Ty przez swe objawienie opromieniłeś nas swoim światłem,
– spraw, abyśmy je dziś nieśli spotykanym ludziom.
Chryste, Ty przyjmując chrzest od swojego sługi, uniżyłeś się, aby dać nam
przykład pokory,
– spraw, byśmy umieli pokornie służyć braciom.
Chryste, Ty przez swój chrzest obmyłeś nas z wszelkiego grzechu i przywróciłeś nam godność dzieci Twojego Ojca,
– udziel ducha przybrania wszystkim, którzy Cię szukają.
Chryste, Ty przez swój chrzest uświęciłeś całe stworzenie i otworzyłeś
ochrzczonym bramę nawrócenia,
– spraw, abyśmy byli w świecie sługami Twojej Ewangelii.
Chryste, Ty przez swój chrzest objawiłeś nam Trójcę Świętą,
– odnów ducha przybranych dzieci w kapłańskim ludzie ochrzczonych.



Ojcze nasz.
Modlitwa:
Boże, Ty chcesz, aby wszyscy ludzie zostali zbawieni i doszli do poznania
prawdy, q wejrzyj na swoje wielkie żniwo i poślij na nie robotników, aby
głosili Ewangelię wszelkiemu stworzeniu, * niech Twój lud zgromadzony
przez Słowo Życia i pokrzepiony mocą sakramentów kroczy drogą zbawienia i miłości. Przez naszego Pana Jezusa Chrystusa, Twojego Syna który
z Tobą żyje i króluje w jedności Ducha Świętego, * Bóg, przez wszystkie
wieki wieków.



\startday[title={Bóg walczy}] % {{{1
Wezwanie: Uwielbiajmy Chrystusa Pana, który dla nas był kuszony
i ukrzyżowany
Hymn: piosenka dnia
Psalmodia
1 ant. Budzą się moje oczy jeszcze przed świtem, * aby rozważać
Twoje słowo, Panie.
Psalm 119,145-152
Z całego serca wołam, wysłuchaj mnie, Panie, *
zachować chcę Twoje ustawy
Wołam do Ciebie, a Ty mnie wybaw, *
będę strzegł Twoich napomnień.
Przychodzę o świcie i wołam, *
pokładam ufność w Twoich słowach.
Budzą się moje oczy jeszcze przed świtem, *
aby rozważać Twoje słowo.
W dobroci swej, Panie, słuchaj głosu mego *
i daj mi życie zgodne z Twoim przykazaniem.
Zbliżają się niegodziwi moi prześladowcy, *
dalecy są oni od Twojego Prawa.
Jesteś blisko, Panie, *
i wszystkie Twe przykazania są prawdą.
Od dawna wiem z Twoich napomnień, *
że ustaliłeś je na wieki.
Chwała Ojcu i Synowi, *
i Duchowi Świętemu
Jak była na początku, teraz i zawsze, *
i na wieki wieków. Amen.
Ant. Budzą się moje oczy jeszcze przed świtem, / aby rozważać Twoje
słowo, Panie.
2 ant. Pan jest moją mocą i źródłem męstwa, * Jemu zawdzięczam
moje ocalenie.


Pieśń (Wj 15, 1b-13.17-18)
Zaśpiewam na cześć Pana, który okrył się sławą, *
gdy konia i jeźdźca pogrążył w morskiej toni.
Pan jest moją mocą i źródłem męstwa, *
Jemu zawdzięczam moje ocalenie.
On Bogiem moim, uwielbiać Go będę, *
On Bogiem mego ojca, będę Go wywyższać.
Pan, wojownik potężny, *
«Ten, który jest», brzmi Jego imię.
Rzucił w morze rydwany faraona i wojska jego, *
wybrani wodzowie legli w Morzu Czerwonym.
Ogarnęły ich przepaści, *
jak głaz runęli w głębinę.
Okazała się wtedy potęga Twej prawicy, Panie, *
prawica Twoja, Panie, zmiażdżyła nieprzyjaciół.
Pełen mocy zniszczyłeś swoich wrogów, *
wybuchnąłeś gniewem, co spalił ich jak słomę.
Od Twojego tchnienia spiętrzyły się wody, q
jak wał stanęły ogromne fale, *
w pośrodku morza zakrzepły otchłanie.
Mówił nieprzyjaciel: „Będę ścigał, dopadnę, q
rozdzielę łupy i nasycę mą duszę, *
dobędę miecza, moja ręka ich zniszczy!”
Lecz Tyś posłał swój wicher i przykryło ich morze, *
zatonęli jak ołów wśród gwałtownych wirów.
Któż spomiędzy bogów równy Tobie, Panie, q
kto blaskiem świętości podobny do Ciebie, *
straszliwy w czynach, sprawiający cuda?
Wyciągnąłeś prawicę *
i wchłonęła ich ziemia.
Ty zaś wiodłeś swą łaską lud oswobodzony q
i doprowadziłeś go Twoją mocą*
do swego mieszkania świętego.
Wprowadziłeś ich i osadziłeś na górze Twojego dziedzictwa, *
w miejscu, któreś uczynił swoim mieszkaniem,
W świątyni zbudowanej Twoimi rękami; *
Pan jest Królem na zawsze i na wieki.
Chwała Ojcu i Synowi, *
i Duchowi Świętemu.
Jak była na początku, teraz i zawsze, *
i na wieki wieków. Amen.


Ant. Pan jest moją mocą i źródłem męstwa, / Jemu zawdzięczam moje
ocalenie.
3 ant. Chwalcie Pana, * wszystkie narody. q
Psalm 117
Chwalcie Pana, wszystkie narody, *
q wysławiajcie Go, wszystkie ludy!
Bo potężna nad nami Jego łaska, *
a wierność Pana trwa na wieki.
Chwała Ojcu i Synowi, *
i Duchowi Świętemu.
Jak była na początku, teraz i zawsze, *
i na wieki wieków. Amen.
Ant. Chwalcie Pana, wszystkie narody.
Czytanie: Wj 14,13-14
Mojżesz odpowiedział ludowi: Nie bójcie się! Pozostańcie na swoim miejscu, a zobaczycie zbawienie od Pana, jakie zgotuje nam dzisiaj. Egipcjan,
których widzicie teraz, nie będziecie już nigdy oglądać. Pan będzie walczył
za was, a wy będziecie spokojni.
Ant. do pieśni Zachariasza:
Duch wyprowadził Jezusa na pustynię, * aby był kuszony przez diabła.
/ A gdy przepościł czterdzieści dni i nocy, / głód potem odczuwał.
Prośby
Błogosławmy naszego Odkupiciela, który łaskawie wysłużył nam ten czas
zbawienia. Z pokorą zanośmy do Niego nasze prośby:
Stwórz w nas, o Panie, nowego ducha.
Chryste, nasze życie, Ty zechciałeś, abyśmy przez chrzest zostali pogrzebani z Tobą w śmierci i z Tobą zostali wskrzeszeni,
– dopomóż nam dzisiaj postępować w nowości życia.
Panie, Ty wszystkim dobrze czyniłeś,
– spraw, abyśmy troszczyli się o wspólne dobro wszystkich ludzi.
Daj, abyśmy zgodnie współpracowali w doczesnej społeczności,
– a zarazem zdążali do wiecznej ojczyzny.



Chryste, lekarzu ciał i dusz, ulecz rany naszego serca,
– abyśmy stale korzystali z Twej pomocy na drodze świętości.
Ojcze nasz.
Modlitwa:
Wszechmogący, wieczny Boże, wejrzyj na naszą słabość w walce z mocami
ciemności * i wyciągnij w naszej obronie swoją potężną prawicę. Przez naszego Pana Jezusa Chrystusa, Twojego Syna, q który z Tobą żyje i króluje
w jedności Ducha Świętego, * Bóg, przez wszystkie wieki wieków.



\startday[title={Plagi}] % {{{1
Wezwanie: Jeśli głos Pana usłyszycie, nie zatwardzajcie serc waszych
Hymn: piosenka dnia
Psalmodia
1 ant. O świcie * napełnia nas, Panie, Twoja łaska.
Psalm 90
Panie, Ty dla nas byłeś ucieczką *
z pokolenia na pokolenie.
Zanim narodziły się góry, q
nim powstał świat i ziemia, *
od wieku po wiek Ty jesteś Bogiem.
Obracasz w proch człowieka, *
i mówisz: „Wracajcie, synowie ludzcy”.
Bo tysiąc lat w Twoich oczach q
jest jak wczorajszy dzień, który minął, *
albo straż nocna.
Porywasz ich, stają się niby sen poranny, *
jak trawa, która rośnie:
Rankiem zielona i kwitnąca, *
wieczorem więdnie i usycha.
Zaprawdę, Twój gniew nas niszczy, *
trwoży nas Twoje oburzenie.
Położyłeś przed sobą nasze grzechy, *
nasze skryte winy w świetle Twojego oblicza.
Wszystkie nasze dni mijają w Twoim gniewie, *
nasze lata dobiegają końca jak westchnienie.
Miarą naszego życia jest lat siedemdziesiąt, *
osiemdziesiąt, gdy jesteśmy mocni.
A większość z nich, to trud i marność, *
bo szybko mijają, my zaś odlatujemy.
Któż może poznać siłę Twego gniewu *
i kto znieść zdoła moc Twego oburzenia?
Naucz nas liczyć dni nasze, *
byśmy zdobyli mądrość serca.


Powróć, Panie, jak długo będziesz zwlekał? *
Bądź litościwy dla sług Twoich.
Nasyć nas o świcie swoją łaską, *
abyśmy przez wszystkie dni nasze mogli się radować i cieszyć.
Daj radość w zamian za dni Twego ucisku, *
za lata, w których zaznaliśmy niedoli.
Niech sługom Twoim ukaże się Twe dzieło, *
a Twoja chwała nad ich synami.
Dobroć Pana, Boga naszego, niech będzie nad nami q
i wspieraj pracę rąk naszych, *
dzieło rąk naszych wspieraj!
Chwała Ojcu i Synowi, *
i Duchowi Świętemu.
Jak była na początku, teraz i zawsze, *
i na wieki wieków. Amen.
Ant. O świcie napełnia nas, Panie, Twoja łaska.
2 ant. Wszystkie krańce ziemi * głoszą chwałę Pana.
Pieśń (Iz 42, 10-16)
Śpiewajcie Panu pieśń nową, *
Jego chwała aż po krańce ziemi.
Niech sławi Go morze i to, co je napełnia, *
i wyspy razem z tymi, którzy tam mieszkają.
Niech woła pustynia i miasta, *
osiedla plemienia Kedar.
Mieszkańcy Sela niech wznoszą okrzyki, *
ze szczytów gór niech wołają radośnie.
Niech Panu oddają chwałę, *
Jego cześć niech głoszą na wyspach.
Jak mocarz Pan kroczy, *
jak wojownik pobudza odwagę;
Rzuca hasło, okrzyk wydaje wojenny, *
nieprzyjaciół męstwem przewyższa.
„Długo milczałem, *
powstrzymywałem siebie w spokoju,
Teraz jak rodząca zakrzyknę, *
będę dyszał gniewem, aż tchu mi zabraknie.
Wypalę góry i pagórki, *
sprawię, że wyschnie cała ich zieleń,


Rzeki w stawy przemienię, *
i osuszę jeziora.
Uczynię, że niewidomi pójdą po nieznanej drodze, *
powiodę ich ścieżkami, których nie znają;
W światło zamienię ich ciemności, *
a miejsca wyboiste w równinę”.
Chwała Ojcu i Synowi, *
i Duchowi Świętemu.
Jak była na początku, teraz i zawsze, *
i na wieki wieków. Amen.
Ant. Wszystkie krańce ziemi głoszą chwałę Pana.
3 ant. Chwalcie imię Pana, * wy, co stoicie w Jego domu.
Psalm 135, 1-12
Chwalcie imię Pana, *
chwalcie, Pańscy słudzy,
Którzy stoicie w domu Pana, *
na dziedzińcach domu Boga naszego.
Chwalcie Pana, bo Pan jest dobry, *
śpiewajcie Jego imieniu, bo jest łaskawy.
Pan bowiem wybrał sobie Jakuba, *
na własność wyłączną wybrał Izraela.
Wiem, że Pan jest wielki, *
że nasz Pan jest nad wszystkimi bogami.
Cokolwiek spodoba się Panu, q
uczyni na niebie i na ziemi, *
na morzu i na wszystkich głębinach.
Z krańców ziemi sprowadza chmury, q
wywołuje deszcz błyskawicami *
i dobywa wiatr ze swoich komór.
Poraził pierworodnych w Egipcie, *
wśród ludzi i wśród zwierząt.
W tobie, kraju egipski, zdziałał znaki i cuda *
przeciw faraonowi i wszystkim jego sługom.
Poraził wiele narodów *
i zgładził potężnych królów:
Amoryckiego króla Sichona i Oga, króla Baszanu, *
i wszystkich królów kananejskich,



A ziemię ich dał w posiadanie, *
w posiadanie Izraela, swego narodu.
Chwała Ojcu i Synowi, *
i Duchowi Świętemu
Jak była na początku, teraz i zawsze, *
i na wieki wieków. Amen.
Ant. Chwalcie imię Pana, / wy, co stoicie w Jego domu.
Czytanie: Wj 4, 21-23
Pan rzekł do Mojżesza: Gdy będziesz zbliżał się do Egiptu, pamiętaj o władzy czynienia wszelkich cudów, jaką ci dałem do ręki, i okaż ją przed faraonem. Ja zaś uczynię upartym jego serce, że nie zechce zezwolić na wyjście
ludu. A ty wtedy powiesz do faraona: To mówi Pan: Synem moim pierworodnym jest Izrael. Mówię ci: Wypuść mojego syna, aby mi cześć oddawał;
bo jeśli zwlekać będziesz z wypuszczeniem go, to Ja ześlę śmierć na twego
syna pierworodnego.
Antyfona do pieśni Zachariasza:
Bóg jest duchem, * potrzeba więc, by Jego czciciele / oddawali Mu cześć
w Duchu i prawdzie.
Prośby
Błogosławmy naszego Odkupiciela, który łaskawie wysłużył nam ten czas
zbawienia. Z pokorą zanośmy do Niego nasze prośby:
Stwórz w nas, o Panie, nowego ducha.
Chryste, nasze życie, Ty zechciałeś, abyśmy przez chrzest zostali pogrzebani z Tobą w śmierci i z Tobą zostali wskrzeszeni,
– dopomóż nam dzisiaj postępować w nowości życia.
Panie, Ty wszystkim dobrze czyniłeś,
– spraw, abyśmy troszczyli się o wspólne dobro wszystkich ludzi.
Daj, abyśmy zgodnie współpracowali w doczesnej społeczności,
– a zarazem zdążali do wiecznej ojczyzny.
Chryste, lekarzu ciał i dusz, ulecz rany naszego serca,
– abyśmy stale korzystali z Twej pomocy na drodze świętości.



Ojcze nasz.
Modlitwa:
Wszechmogący Boże, wejrzyj na zgromadzenie Twoich wiernych q i spraw,
aby nasze dusze oczyszczone przez umartwienie ciała * jaśniały pragnieniem posiadania Ciebie. Przez naszego Pana Jezusa Chrystusa Twojego
Syna, q który z Tobą żyje i króluje w jedności Ducha Świętego, * Bóg, przez
wszystkie wieki wieków.



\startday[title={Obłok}] % {{{1
Wezwanie: Uwielbiajmy Pana, * którego oblicze jaśnieje nad nami
Hymn: piosenka dnia
Psalmodia
1 ant. Boże, Twoja droga jest święta, * nikt nie dorówna wielkością
naszemu Bogu.
Psalm 77
Głos mój się wznosi do Boga, gdy wołam, *
głos mój wznoszę do Boga, aby mnie usłyszał.
W dniu mej niedoli szukam Pana, *
w nocy niestrudzenie wyciągam rękę.
Dusza moja jest niepocieszona, q
jęczę, kiedy wspomnę Boga, *
słabnie mój duch, gdy rozmyślam.
Ty spędzasz sen z moich powiek, *
z niepokoju mówić nie potrafię.
Rozpamiętuję dni, które dawno minęły, *
i lata poprzednie wspominam.
Rozmyślam nocą w swym sercu, *
rozważam, a duch mój docieka:
„Czy Bóg odrzuci na wieki *
i już nie będzie łaskawy?
Czy Jego łaskawość ustała na zawsze *
i słowo zamilkło na pokolenia?
Czy Bóg zapomniał o litości, *
czy w gniewie powstrzymał swe miłosierdzie?”
I mówię: „Jakże to bolesne, *
że odwróciła się ode mnie prawica Najwyższego”.
Wspominam dzieła Pana, *
oto wspominam Twoje dawne cuda.
Rozmyślam o wszystkich Twych dziełach *
i czyny Twoje wspominam.
Boże, Twoja droga jest święta, *
który z bogów dorówna wielkością naszemu Bogu?


Ty jesteś Bogiem działającym cuda, *
ludziom objawiłeś swą potęgę.
Ramieniem swoim Twój lud wybawiłeś, *
synów Jakuba i Józefa.
Boże, ujrzały Cię wody, q
ujrzały Cię wody i zadrżały, *
wzburzyły się ich odmęty.
Chmury wylały wody, q
zahuczały chmury *
i Twoje strzały się posypały.
Głos Twego grzmotu jak łoskot wozu, q
pioruny świat rozjaśniły, *
ziemia poruszyła się i zatrzęsła.
Twoja droga wiodła przez wody, q
Twoja ścieżka przez wodne obszary *
i nie znać było Twych śladów.
Wiodłeś Twój lud jak trzodę *
ręką. Mojżesza i Aarona.
Chwała Ojcu i Synowi, *
i Duchowi Świętemu
Jak była na początku, teraz i zawsze, *
i na wieki wieków. Amen.
Ant. Boże, Twoja droga jest święta, / nikt nie dorówna wielkością Bogu.
2 ant. Moje serce raduje się w Panu, * który poniża i wywyższa.
Pieśń (1 Sm 2, 1-10)
Moje serce raduje się w Panu, *
dzięki Niemu moc moja wzrasta.
Szeroko otwarłam usta przeciw moim wrogom, *
bo cieszyć się mogę Twoją pomocą.
Nikt nie jest tak święty jak Ty, Panie, q
poza Tobą bowiem nie ma nikogo, *
prócz naszego Boga nie ma innej ostoi.
Nie powtarzajcie słów pełnych pychy, *
niech mowa harda z ust waszych nie wychodzi,
Gdyż Pan jest Bogiem wszechwiedzącym *
i On ocenia uczynki.



Łuk potężnych się łamie, *
a mocą przepasują się słabi.
Syci za chleb się najmują, *
głodni zaś odpoczywają.
Niepłodna rodzi siedmioro, *
a matka wielu dzieci usycha.
Pan daje śmierć i życie, *
wtrąca do Otchłani i z niej wyprowadza.
Pan czyni ubogim lub bogatym, *
poniża i wywyższa.
Biedaka z prochu podnosi, *
z błota dźwiga nędzarza,
By go wśród książąt posadzić *
i dać mu tron chwały.
Fundamenty ziemi należą do Pana *
i na nich świat On położył.
On strzeże kroków swoich wiernych, q
grzesznicy zaś zginą w ciemnościach, *
bo nie własną siłą człowiek zwycięża.
Pan wniwecz opornych obraca *
i przeciw nim grzmi na niebiosach.
Pan sądzi krańce ziemi, q
króla obdarza potęgą *
i wywyższa moc swego pomazańca.
Chwała Ojcu i Synowi, *
i Duchowi Świętemu.
Jak była na początku, teraz i zawsze, *
i na wieki wieków. Amen.
Ant. Moje serce raduje się w Panu, / który poniża i wywyższa.
3 ant. Pan króluje, * wesel się, ziemio. q
Psalm 97
Pan króluje, wesel się, ziemio, *
q radujcie się, liczne wyspy!
Obłok i ciemność wokół Niego, *
prawo i sprawiedliwość podstawą Jego tronu.
Przed Jego obliczem idzie ogień *
i dokoła pożera nieprzyjaciół Jego.



Jego błyskawice wszechświat rozświetlają, *
a ziemia drży na ten widok.
Góry jak wosk topnieją przed obliczem Pana, *
przed obliczem Władcy całej ziemi.
Jego sprawiedliwość rozgłaszają niebiosa *
i wszystkie ludy widzą Jego chwałę.
Niech zawstydzą się wszyscy, którzy czczą posągi q
i chlubią się bożkami. *
Niech wszystkie bóstwa hołd Mu oddają!
Słyszy o tym i cieszy się Syjon, q
radują się miasta Judy *
z Twoich wyroków, o Panie.
Ponad całą ziemią Tyś bowiem wywyższony *
i nieskończenie wyższy od wszystkich bogów.
Pan tych miłuje, którzy zła nienawidzą, q
On strzeże dusz świętych swoich, *
wydziera je z rąk grzeszników.
Światło wschodzi dla sprawiedliwego *
i radość dla ludzi prawego serca.
Weselcie się w Panu, sprawiedliwi, *
i sławcie Jego święte imię.
Chwała Ojcu i Synowi, *
i Duchowi Świętemu.
Jak była na początku, teraz i zawsze, *
i na wieki wieków. Amen.
Ant. Pan króluje, wesel się, ziemio.
Czytanie: Lb 9, 15-16
W dniu, kiedy ustawiono przybytek, okrył go wraz z Namiotem Świadectwa obłok, i od wieczora aż do rana pozostawał nad przybytkiem na kształt
ognia. I tak działo się zawsze: obłok okrywał go w dzień, a w nocy – jakby
blask ognia.
Antyfona do pieśni Zachariasza:
Nigdy nie słyszano, * aby ktoś przywrócił wzrok niewidomemu od urodzenia; / tylko Chrystus, Syn Boży, tego dokonał.



Prośby:
Uwielbiajmy Boga nieskończonej dobroci i błagajmy Go przez Jezusa Chrystusa, który zawsze żyje, aby się wstawiać za nami:
Zapal w nas Panie, ogień swej miłości.
Spraw, miłosierny Boże, abyśmy pełnili dzisiaj liczne dzieła miłości
– i okazywali życzliwość wszystkim naszym bliźnim.
Ty w czasie potopu uratowałeś Noego przez arkę,
– wybaw katechumenów przez wodę chrztu świętego.
Daj, abyśmy nie poprzestawali na karmieniu się tylko chlebem,
– ale szukali pożywienia w słowie, które pochodzi z ust Twoich.
Pomóż nam usuwać spośród nas wszelką niezgodę,
– daj, abyśmy się radowali pokojem i miłością.
Ojcze nasz.
Modlitwa:
Boże, nasz Ojcze, przez paschalne misterium swojego Syna dokonałeś naszego odkupienia, q dlatego w sakramentalnych znakach głosimy śmierć
i zmartwychwstanie Chrystusa, * spraw, abyśmy stale doznawali wzrostu
Twojej łaski. Przez naszegp Pana Jezusa Chrystusa, Twojego Syna, q który
z Tobą żyje i króluje w jedności Ducha Świętego, * Bóg, przez wszystkie
wieki wieków.



\startday[title={Baranek}] % {{{1
Wezwanie: Uwielbiajmy Chrystusa Króla, * który dla nas został
podwyższony na krzyżu.
Hymn: piosenka dnia
Psalmodia
1 ant. Pokornym i skruszonym sercem * Ty, Boże, nie gardzisz.
Psalm 51
Zmiłuj się nade mną, Boże, w łaskawości swojej, *
w ogromie swej litości zgładź nieprawość moją.
Obmyj mnie zupełnie z mojej winy *
i oczyść mnie z grzechu mojego.
Uznaję bowiem nieprawość moją, *
a grzech mój jest zawsze przede mną.
Przeciwko Tobie samemu zgrzeszyłem *
i uczyniłem, co złe jest przed Tobą,
Abyś okazał się sprawiedliwy w swym wyroku *
i prawy w swoim sądzie.
Oto urodziłem się obciążony winą *
i jako grzesznika poczęła mnie matka.
A Ty masz upodobanie w ukrytej prawdzie, *
naucz mnie tajemnic mądrości.
Pokrop mnie hizopem, a stanę się czysty, *
obmyj mnie, a nad śnieg wybieleję.
Spraw, abym usłyszał radość i wesele, *
niech się radują kości, które skruszyłeś.
Odwróć swe oblicze od moich grzechów *
i zmaż wszystkie moje przewinienia.
Stwórz, Boże, we mnie serce czyste *
i odnów we mnie moc ducha.
Nie odrzucaj mnie od swego oblicza *
i nie odbieraj mi świętego ducha swego.
Przywróć mi radość Twojego zbawienia *
i wzmocnij mnie duchem ofiarnym



Będę nieprawych nauczał dróg Twoich *
i wrócą do Ciebie grzesznicy.
Uwolnij mnie, Boże, od kary za krew przelaną, q
Boże, mój Zbawco, *
niech sławi mój język sprawiedliwość Twoją.
Panie, otwórz wargi moje, *
a usta moje będą głosić Twoją chwałę.
Ofiarą bowiem Ty się nie radujesz *
a całopalenia, choćbym dał, nie przyjmiesz.
Boże, moją ofiarą jest duch skruszony, *
pokornym i skruszonym sercem Ty, Boże, nie gardzisz.
Panie, okaż Syjonowi łaskę w Twej dobroci, *
odbuduj mury Jeruzalem.
Wtedy przyjmiesz prawe ofiary: dary i całopalenia, *
wtedy składać będą cielce na Twoim ołtarzu.
Chwała Ojcu i Synowi, *
i Duchowi Świętemu.
Jak była na początku, teraz i zawsze, *
i na wieki wieków. Amen.
Ant. Pokornym i skruszonym sercem / Ty, Boże, nie gardzisz.
2 ant. Gdy się gniewasz, Panie, * wspomnij na swe miłosierdzie.
Pieśń (Ha 3, 2-4. 13a. 15-19)
Usłyszałem, Panie, Twoje orędzie, *
zobaczyłem, Panie, Twoje dzieło.
Gdy czas nadejdzie, niech ono odżyje, q
pozwól nam je poznać, gdy zbliży się pora, *
w gniewnym zapale wspomnij na swą litość!
Bóg przychodzi z Temanu, *
Święty z góry Paran.
Jego majestat okrywa niebiosa, *
a Jego chwały pełna jest ziemia.
Jego wspaniałość podobna do światła, q
z Jego rąk tryskają promienie, *
moc Jego w nich jest ukryta.
Wyszedłeś, aby lud swój ocalić *
i wybawić Twego pomazańca.



Konie bezbożnika wdeptałeś w morze, *
w kipiącą topiel wód ogromnych.
Usłyszałem, i me serce struchlało, *
na ten głos moje wargi zadrżały,
Moje kości przeniknęła trwoga, *
zachwiały się moje kroki.
Jednak w spokoju czekam na klęskę, *
która spotka lud naszych gnębicieli.
Choć drzewo figowe nie rozwija pąków *
i winnice nie wydają plonów,
Chociaż zawiodły zbiory oliwek, *
a pola nie przynoszą żywności,
Choć stada owiec znikają z owczarni *
i nie ma wołów w zagrodach,
Ja się jednak rozraduję w Panu *
i rozweselę w Bogu, moim Zbawicielu.
Pan, który jest moją siłą, q
uczyni me nogi jak nogi jelenia *
i na wyżyny mnie wyprowadzi.
Chwała Ojcu i Synowi, *
i Duchowi Świętemu.
Jak była na początku, teraz i zawsze, *
i na wieki wieków. Amen.
Ant. Gdy się gniewasz, Panie, / wspomnij na swe miłosierdzie.
3 ant. Chwal, Jeruzalem, * Pana. q
Psalm 147 B
Chwal, Jeruzalem, Pana, *
q wysławiaj twego Boga, Syjonie!
Umacnia bowiem zawory bram twoich *
i błogosławi synom twoim w tobie.
Zapewnia pokój twoim granicom *
i wyborną pszenicą ciebie darzy.
Zsyła na ziemię swoje polecenia, *
a szybko mknie Jego słowo.
On prószy śniegiem jak wełną *
i szron jak popiół rozsypuje.



On grad rozrzuca jak okruchy chleba, *
od Jego mrozu ścinają się wody.
Posyła słowo, i lód topnieje, *
powieje wiatrem, i rzeki płyną.
Oznajmił swoje słowo Jakubowi, *
Izraelowi ustawy swe i wyroki
Nie uczynił tego dla innych narodów, *
nie oznajmił im swoich wyroków.
Chwała Ojcu i Synowi, *
i Duchowi Świętemu.
Jak była na początku, teraz i zawsze, *
i na wieki wieków. Amen.
Ant. Chwal, Jeruzalem, Pana.
Czytanie: Wj 12, 21-23
Mojżesz zwołał wszystkich starszych Izraela i rzekł do nich: Odłączcie i weźcie baranka dla waszych rodzin i zabijcie jako paschę. Weźcie gałązkę hizopu i zanurzcie ją we krwi, która jest w naczyniu, i krwią z naczynia skropcie
próg i oba odrzwia. Aż do rana nie powinien nikt z was wychodzić przed
drzwi swego domu. A gdy Pan będzie przechodził, aby porazić Egipcjan,
a zobaczy krew na progu i na odrzwiach, to ominie Pan takie drzwi i nie
pozwoli Niszczycielowi wejść do tych domów, aby was zabijał.
Antyfona do pieśni Zachariasza:
Łazarz, przyjaciel nasz, zasnął, * lecz idę, aby go obudzić.
Prośby
Błogosławmy naszego Odkupiciela, który łaskawie wysłużył nam ten czas
zbawienia. Z pokorą zanośmy do Niego nasze prośby:
Stwórz w nas, o Panie, nowego ducha.
Chryste, nasze życie, Ty zechciałeś, abyśmy przez chrzest zostali pogrzebani z Tobą w śmierci i z Tobą zostali wskrzeszeni,
– dopomóż nam dzisiaj postępować w nowości życia.
Panie, Ty wszystkim dobrze czyniłeś,
– spraw, abyśmy troszczyli się o wspólne dobro wszystkich ludzi.
Daj, abyśmy zgodnie współpracowali w doczesnej społeczności,
– a zarazem zdążali do wiecznej ojczyzny.



Chryste, lekarzu ciał i dusz, ulecz rany naszego serca,
– abyśmy stale korzystali z Twej pomocy na drodze świętości.
Ojcze nasz.
Modlitwa:
Prosimy Cię, Panie, nasz Boże, q udziel nam łaski, abyśmy gorliwie naśladowali miłość Twojego Syna, * który oddał własne życie za zbawienie
świata. Przez naszego Pana, Jezusa Chrystusa, Twojego Syna, q który
z Tobą żyje i króluje w jedności Ducha Świętego, * Bóg, przez wszystkie
wieki wieków.



\startday[title={Pascha}] % {{{1
Wezwanie: Uwielbiajmy Chrystusa Syna Bożego, * który nas odkupił
krwią swoją
Hymn: piosenka dnia
Psalmodia
1 ant. Chwalcie Pana, * bo Jego łaska na wieki.
Psalm 136
I
Chwalcie Pana, bo jest dobry, *
bo Jego łaska na wieki.
Chwalcie Boga nad bogami, *
bo Jego łaska na wieki.
Chwalcie Pana nad panami, *
bo Jego łaska na wieki.
On sam cudów wielkich dokonał, *
bo Jego laska na wieki.
On w swej mądrości uczynił niebiosa, *
bo Jego łaska na wieki.
On rozpostarł ziemię nad wodami, *
bo Jego łaska na wieki.
On uczynił światła ogromne, *
bo Jego łaska na wieki.
Słońce, by dniem władało, *
bo Jego łaska na wieki.
Księżyc i gwiazdy, by władały nocą, *
bo Jego łaska na wieki
Chwała Ojcu i Synowi, *
i Duchowi Świętemu.
Jak była na początku, teraz i zawsze, *
i na wieki wieków. Amen.
Ant. Chwalcie Pana, bo Jego łaska na wieki.



2 ant. Wielkie i godne podziwu * są dzieła Twoje, / Panie, Boże
wszechmogący.
II
On Egipcjanom pobił pierworodnych, *
bo Jego łaska na wieki.
I wywiódł spośród nich Izraela, *
bo Jego łaska na wieki.
Ręką potężną, wyciągniętym ramieniem, *
bo Jego łaska na wieki.
Rozdzielił Morze Czerwone, *
bo Jego łaska na wieki.
I środkiem morza przeprowadził Izraela, *
bo Jego łaska na wieki.
Faraona z wojskiem strącił w Morze Czerwone, *
bo Jego łaska na wieki.
I prowadził swój lud przez pustynię, *
bo Jego łaska na wieki.
On pobił wielkich królów, *
bo Jego łaska na wieki.
On uśmiercił królów potężnych, *
bo Jego łaska na wieki.
Sichona, króla Amorytów, *
bo Jego łaska na wieki.
I Oga, króla Baszanu, *
bo Jego łaska na wieki.
A ziemię ich dał na własność, *
bo Jego łaska na wieki.
Na własność swemu słudze Izraelowi, *
bo Jego łaska na wieki
Pamiętał o nas w naszym poniżeniu, *
bo Jego łaska na wieki.
I uwolnił nas od wrogów, *
bo Jego łaska na wieki.
On pokarm daje każdemu ciału, *
bo Jego łaska na wieki.
Chwalcie Boga, niebiosa, *
bo Jego łaska na wieki.
Chwała Ojcu i Synowi, *
i Duchowi Świętemu.



Jak była na początku, teraz i zawsze, *
i na wieki wieków. Amen.
Ant. Wielkie i godne podziwu są dzieła Twoje, / Panie, Boże
wszechmogący.
3 ant. Pan mówi / Błogosławieństwem mój lud się nasyci.
Pieśń (Jr 31, 10-14)
Słuchajcie, narody, słowa Pańskiego, *
głoście je na wyspach odległych i mówcie:
„Ten, który rozproszył Izraela, znów go zgromadzi *
i będzie nad nim czuwał jak pasterz nad swym stadem”.
Pan bowiem uwolni Jakuba, *
wybawi go z ręki silniejszych od niego.
Przyjdą z weselem na szczyt Syjonu q
i rozradują się błogosławieństwem Pana: *
zbożem, winem, oliwą, owcami i wołami.
Ich życie stanie się podobne do zroszonego ogrodu *
i już nigdy więcej sił im nie zabraknie.
Wtedy dziewica rozweseli się w tańcu, *
a młodzieńcy i starcy cieszyć się będą. *
Smutek ich bowiem w radość zamienię, *
pocieszę i rozweselę po ich troskach.
Tłuszczem z ofiar obficie obdarzę kapłanów, *
i błogosławieństwem mój lud się nasyci.
Chwała Ojcu i Synowi, *
i Duchowi Świętemu.
Jak była na początku, teraz i zawsze, *
i na wieki wieków. Amen.
Ant. Pan mówi: / Błogosławieństwem mój lud się nasyci.
Czytanie: Wj 12,11-14
Tak zaś spożywać będziecie baranka: Biodra wasze będą przepasane, sandały na waszych nogach i laska w waszym ręku. Spożywać będziecie pośpiesznie, gdyż jest to Pascha na cześć Pana. Tej nocy przejdę przez Egipt,
zabiję wszystko pierworodne w ziemi egipskiej od człowieka aż do bydła
i odbędę sąd nad wszystkimi bogami Egiptu – Ja, Pan. Krew będzie wam
służyła do oznaczenia domów, w których będziecie przebywać. Gdy ujrzę



krew, przejdę obok i nie będzie pośród was plagi niszczycielskiej, gdy będę
karał ziemię egipską. Dzień ten będzie dla was dniem pamiętnym i obchodzić go będziecie jako święto dla uczczenia Pana. Po wszystkie pokolenia
– na zawsze w tym dniu świętować będziecie.
Antyfona do pieśni Zachariasza:
Gorąco pragnąłem * spożyć tę Paschę z wami, / zanim będę cierpiał.
Prośby
Zanośmy pokorne błagania do Chrystusa, Wiecznego Kapłana, którego Ojciec namaścił Duchem Świętym, aby więźniom głosił wyzwolenie:
Panie, zmiłuj się nad nami.
Panie, ty udałeś się do Jerozolimy, aby podjąć mękę i wejść do chwały,
– doprowadź swój Kościół do wiekuistej Paschy.
Chryste, gdy byłeś podwyższony na krzyżu, Twój bok został przebity włócznią żołnierza,
– ulecz nasze rany.
Panie, Twój krzyż stał się drzewem życia,
– udziel jego owoców odrodzonym przez chrzest święty.
Chryste, Ty wisząc na drzewie krzyża, przebaczyłeś pokutującemu łotrowi,
– odpuść nam, grzesznym, nasze winy.
Ojcze nasz.
Modlitwa:
Boże, Ty przez mękę Chrystusa, Twojego Syna a naszego Pana, zniweczyłeś śmierć, która wynikła z grzechu pierworodnego i jako dziedzictwo
przeszła na wszystkie pokolenia, q spraw, abyśmy wszczepieni w Chrystusa i uświęceni przez łaskę nosili w sobie podobieństwo do Niego, * jak
z konieczności natury nosiliśmy podobieństwo do Adama. Przez naszego
Pana Jezusa Chrystusa, Twojego Syna, q który z Tobą żyje i króluje w jedności Ducha Świętego, * Bóg, przez wszystkie wieki wieków.


\startday[title={Zwycięstwo}] % {{{1
\startday[title={Manna}] % {{{1
Wezwanie: Alleluja, uwielbiajmy Chrystusa Pana, * który wstępuje
do nieba, / alleluja
Hymn: piosenka dnia
Psalmodia
1 ant. Wzbudź swą potęgę, Panie, * i przyjdź nam z pomocą.
Psalm 80
Usłysz, Pasterzu Izraela, q
Ty, który jak trzodę prowadzisz ród Józefa, *
Ty, który zasiadasz nad cherubami!
Ukaż się przed Efraimem, Beniaminem i Manassesem, *
wzbudź swą potęgę i przyjdź nam z pomocą.
Odnów nas, Boże, q
i rozjaśnij nad nami swoje oblicze, *
a będziemy zbawieni.
Panie, Boże Zastępów, jak długo gniewać się będziesz *
pomimo modłów Twojego ludu?
Nakarmiłeś go chlebem płaczu, *
obficie napoiłeś łzami.
Uczyniłeś nas przyczyną zwady sąsiadów, *
a wrogowie nasi z nas szydzą.
Odnów nas, Boże Zastępów, q
i rozjaśnij nad nami swoje oblicze, *
a będziemy zbawieni.
Przeniosłeś winorośl z Egiptu *
i zasadziłeś ją wygnawszy pogan.
Przygotowałeś dla niej glebę, *
a ona zapuściła korzenie i napełniła ziemię.
W jej cieniu skryły się góry, *
jej gałęzie okryły potężne cedry.
Rozpostarła swe pędy aż do Morza, *
aż do Rzeki swoje latorośle.
Dlaczego zburzyłeś jej ogrodzenie *
i każdy przechodzień zrywa jej grona?


Niszczy ją dzik leśny *
i obgryzają polne zwierzęta.
Powróć, Boże Zastępów, *
wejrzyj z nieba, spójrz i nawiedź tę winorośl.
Chroń to, co zasadziła Twoja prawica, *
latorośl, którą umocniłeś dla siebie.
A ci, którzy ją spalili i wycięli, *
niech zginą od grozy Twojego oblicza.
Wyciągnij rękę nad mężem Twej prawicy,
nad synem człowieczym, którego umocniłeś w swej służbie.
Już więcej nie odwrócimy się od Ciebie, *
daj nam nowe życie, a będziemy Cię chwalili.
Odnów nas, Panie, Boże Zastępów, q
i rozjaśnij nad nami swoje oblicze, *
a będziemy zbawieni.
Chwała Ojcu i Synowi, *
i Duchowi Świętemu.
Jak była na początku, teraz i zawsze, *
i na wieki wieków. Amen.
Ant. Wzbudź swą potęgę, Panie, / i przyjdź nam z pomocą.
2 ant. Pan czynów wspaniałych dokonał * i cała ziemia niech
o tym się dowie.
Pieśń (Iz 12, 1-6)
Wysławiam Cię, Panie, *
bo rozgniewałeś się na mnie,
Lecz Twój gniew się uśmierzył *
i dałeś mi pociechę.
Oto Bóg jest moim zbawieniem, *
Jemu zaufam i bać się nie będę.
Pan jest moją pieśnią i mocą *
i On się stał moim zbawieniem.
Wy zaś z weselem czerpać będziecie *
wodę ze zdrojów zbawienia.
Jeszcze w owym dniu powiecie: *
Chwalcie Pana, wzywajcie Jego imienia!
Ukażcie narodom Jego dzieła, *
przypominajcie, że Jego imię jest chwalebne.



Śpiewajcie Panu, bo czynów wspaniałych dokonał *
i cała ziemia niech o tym się dowie.
Wznoś okrzyki i wołaj radośnie, mieszkanko Syjonu, *
bo wielki jest wśród ciebie Święty Izraela.
Chwała Ojcu i Synowi, *
i Duchowi Świętemu.
Jak była na początku, teraz i zawsze, *
i na wieki wieków. Amen.
Ant. Pan czynów wspaniałych dokonał / i cała ziemia niech
o tym się dowie.
3 ant. Radośnie śpiewajcie Bogu, * który jest naszą mocą. q
Psalm 81
Radośnie śpiewajcie Bogu, który jest naszą mocą, *
q wykrzykujcie na cześć Boga Jakuba!
Zacznijcie śpiewać i w bęben uderzcie, *
w cytrę słodko dźwięczącą i lirę.
Zadmijcie w róg w czas nowiu, *
w czas pełni, w nasz dzień uroczysty.
Bo tak ustanowiono w Izraelu *
przykazania Boga Jakuba.
Ustanowił to prawo dla Józefa, *
gdy wyruszał z ziemi egipskiej.
Słyszę słowa nieznane: q
„Uwolniłem od brzemienia jego barki, *
jego ręce porzuciły kosze.
Wołałeś w ucisku, a Ja cię ocaliłem, q
odpowiedziałem ci z grzmiącej chmury, *
doświadczyłem cię przy wodach Meriba.
Słuchaj, mój ludu, chcę cię napomnieć; *
obyś Mnie posłuchał, Izraelu!
Nie będziesz miał obcego boga, *
cudzemu bogu nie będziesz się kłaniał.
Jam jest Pan, Bóg twój, q
który cię wywiódł z ziemi egipskiej, *
szeroko otwórz usta, abym je napełnił.
Lecz mój lud nie posłuchał mego głosu, *
Izrael nie był Mi posłuszny.



Zostawiłem ich przeto ich twardym sercom, *
niech postępują według swych zamysłów.
Gdyby mój lud Mnie posłuchał, *
a Izrael kroczył moimi drogami,
Natychmiast bym zgniótł ich wrogów *
i obrócił rękę na ich przeciwników.
Schlebialiby Panu ci, którzy Go nienawidzą, *
a kara ich trwałaby na wieki.
A jego bym karmił wyborną pszenicą *
i sycił miodem z opoki”.
Chwała Ojcu i Synowi, *
i Duchowi Świętemu.
Jak była na początku, teraz i zawsze, *
i na wieki wieków. Amen.
Ant. Radośnie śpiewajcie Bogu, / który jest naszą mocą.
Czytanie: Mdr 16, 20
Lud swój żywiłeś pokarmem anielskim i dałeś im bez ich wysiłków gotowy
chleb z nieba, zdolny dać wszelką rozkosz i wszelki smak zaspokoić.
Antyfona do pieśni Zachariasza:
Wstępuję do Ojca mojego i Ojca waszego, * Boga mojego i Boga waszego.
Prośby
Nasz Pan, Jezus Chrystus, wywyższony nad ziemię, wszystkich pociągnął
do siebie. Wzywajmy Go z radością i wołajmy:
Tyś Królem chwały, Jezu Chryste.
Panie Jezu Chryste, Królu chwały, Ty raz ofiarowałeś się za ludzkie grzechy
i jako zwycięzca wstąpiłeś na prawicę Ojca,
– doprowadź zbawionych do pełnej świętości.
Wieczny Kapłanie i Pośredniku Nowego Przymierza, Ty zawsze żyjesz, aby
się wstawiać za nami,
– zbaw lud, który do Ciebie zanosi błaganie.
Ty po swej męce ukazałeś się żywy i przez czterdzieści dni objawiałeś się
uczniom,
– umocnij dzisiaj naszą wiarę.



Ty w dniu dzisiejszym obiecałeś Apostołom zesłać Ducha Świętego, aby
byli Twoimi świadkami aż po krańce ziemi,
– mocą Ducha utwierdzaj nasze świadectwo.
Ojcze nasz.
Modlitwa:
Wszechmogący Boże, wierzymy, że nasz Zbawiciel zasiada w chwale po
Twojej prawicy, q wysłuchaj nasze błagania * i spraw, abyśmy zgodnie
z Jego obietnicą odczuwali, że pozostaje z nami aż do skończenia świata. Przez naszego Pana Jezusa Chrystusa, Twojego Syna, q który z Tobą
żyje i króluje w jedności Ducha Świętego, * Bóg, przez wszystkie wieki
wieków.


\startday[title={Przymierze}] % {{{1
\startday[title={Ziemia obiecana}] % {{{1
Wezwanie: Uwielbiajmy Pana, * który otworzył nam bramy nieba
Hymn: piosenka dnia
Psalmodia
1 ant. Szczęśliwi, którzy mieszkają * w domu Twoim, Panie.
Psalm 84
Jak miłe są Twoje przybytki, *
Panie Zastępów!
Dusza moja stęskniona pragnie przedsionków Pańskich, *
serce moje i ciało radośnie wołają do Boga żywego.
Nawet wróbel znajduje swój dom, a jaskółka gniazdo, q
gdzie złoży swe pisklęta: *
przy ołtarzach Twoich, Panie Zastępów, Królu mój i Boże!
Szczęśliwi, którzy mieszkają w domu Twoim, Panie, *
nieustannie wielbiąc Ciebie.
Szczęśliwi, których moc jest w Tobie, *
którzy zachowują ufność w swym sercu.
Przechodząc suchą doliną, w źródła ją zamieniają, *
a wczesny deszcz błogosławieństwem ją okryje.
Mocy im będzie przybywać, *
ujrzą na Syjonie Boga nad bogami
Panie Zastępów, usłysz modlitwę moją, *
nakłoń ucho, Boże Jakuba.
Spójrz, Boże, tarczo nasza, *
wejrzyj na twarz Twojego pomazańca.
Doprawdy, dzień jeden w przybytkach Twoich *
lepszy jest niż innych tysiące.
Wolę stać w progu domu mojego Boga, *
niż mieszkać w namiotach grzeszników.
Bo Pan Bóg jest słońcem i tarczą, q
On hojnie darzy łaską i chwałą, *
nie odmawia dobrodziejstw żyjącym nienagannie.
Panie Zastępów, *
szczęśliwy człowiek, który ufa Tobie!


Chwała Ojcu i Synowi, *
i Duchowi Świętemu.
Jak była na początku, teraz i zawsze, *
i na wieki wieków. Amen.
Ant. Szczęśliwi, którzy mieszkają w domu Twoim, Panie.
2 ant. Chodźcie, wejdźmy na górę Pana, * do świątyni naszego Boga.
Pieśń (Iz 2, 2-5)
Stanie się na końcu czasów, *
że góra świątyni Pańskiej
Mocno osiądzie na górskich szczytach *
i ponad pagórki się wzniesie.
Nadciągną do niej wszystkie narody, *
liczne plemiona pójdą wołając:
„Chodźcie, wejdźmy na górę Pana, *
do świątyni Boga Jakuba!
Niech On nas pouczy o swoich drogach, *
byśmy kroczyli Jego ścieżkami,
Bo Prawo wyjdzie ze Syjonu *
i z Jeruzalem słowo Pana”.
On będzie rozjemcą pomiędzy ludami, *
osądzi sprawy rozlicznych narodów.
Wtedy swe miecze przekują na pługi, *
a włócznie swoje na sierpy.
Naród przeciw narodowi nie wzniesie już oręża *
i nie będą się więcej ćwiczyć do wojny.
Chodźcie, domu Jakuba, *
postępujmy w światłości Pana!
Chwała Ojcu i Synowi, *
i Duchowi Świętemu.
Jak była na początku, teraz i zawsze, *
i na wieki wieków. Amen.
Ant. Chodźcie, wejdźmy na górę Pana, / do świątyni naszego Boga.
3 ant. Śpiewajcie Panu, * sławcie Jego imię.



Psalm 96
Śpiewajcie Panu pieśń nową, *
śpiewaj Panu, ziemio cała.
Śpiewajcie Panu, sławcie Jego imię, *
każdego dnia głoście Jego zbawienie.
Głoście Jego chwałę wśród wszystkich narodów, *
rozgłaszajcie cuda Jego pośród wszystkich ludów.
Wielki jest Pan, godzien wszelkiej chwały, *
budzi trwogę najwyższą, większą niż inni bogowie.
Bo wszyscy bogowie pogan są tylko ułudą, *
Pan zaś stworzył niebiosa.
Przed Nim kroczą majestat i piękno, *
a potęga i blask w Jego przybytku.
Oddajcie Panu, rodziny narodów, *
oddajcie Panu chwałę i uznajcie Jego potęgę.
Oddajcie Panu chwałę należną Jego imieniu, *
przynieście dary i wejdźcie na Jego dziedzińce.
Uwielbiajcie Pana w świętym przybytku; *
zadrżyj, ziemio cała, przed Jego obliczem.
Głoście wśród ludów, że Pan jest królem, q
On świat tak utwierdził, że się nie zachwieje, *
będzie sprawiedliwie sądził ludy.
Niech się radują niebiosa i ziemia weseli, *
niech szumi morze i wszystko, co je napełnia.
Niech się cieszą pola i wszystko, co na nich rośnie, *
niech wszystkie drzewa w lasach wykrzykują z radości.
Przed obliczem Pana, który już się zbliża, *
który już się zbliża, by osądzić ziemię.
On będzie świat sądził sprawiedliwie, *
a ludy według swej prawdy.
Chwała Ojcu i Synowi, *
i Duchowi Świętemu.
Jak była na początku, teraz i zawsze, *
i na wieki wieków. Amen.
Ant. Śpiewajcie Panu, sławcie Jego imię.



Czytanie: Ap 11,19 i 12,1
Potem Świątynia Boga w niebie się otwarła, i Arka Jego Przymierza ukazała
się w Jego Świątyni, a nastąpiły błyskawice, głosy, gromy, trzęsienie ziemi
i wielki grad. Potem wielki znak się ukazał na niebie: Niewiasta obleczona
w słońce i księżyc pod jej stopami, a na jej głowie wieniec z gwiazd dwunastu.
Antyfona do pieśni Zachariasza:
Piękna i pełna blasku jesteś, Maryjo, * jak jutrzenka wznosisz się do nieba.
Prośby
Oddając cześć naszemu Zbawicielowi, który narodził się z Maryi Dziewicy,
zanośmy do Niego pokorne błagania:
Niech Wniebowzięta wstawia się za nami
Jezu, Słowo odwieczne, Ty wybrałeś sobie Niepokalaną Maryję na mieszkanie,
– wybaw nas od zepsucia grzechu.
Nasz Odkupicielu, Ty sprawiłeś, że Maryja Dziewica stała się godnym dla
Ciebie mieszkaniem i przybytkiem Ducha Świętego,
– daj, abyśmy byli na wieki świątynią Twego Ducha.
Królu wieków, Ty przyjąłeś swą Matkę z ciałem i duszą do niebieskiej chwały,
– spraw, abyśmy zawsze myślą przebywali w niebie.
Panie nieba i ziemi, Ty zechciałeś, aby Maryja jako Królowa stała po Twojej
prawicy,
– daj, abyśmy zasłużyli na udział w Jej chwale.
Ojcze nasz.
Modlitwa:
Boże, Ty wejrzałeś na pokorę Najświętszej Maryi Panny i wyniosłeś Ją do
godności Matki Twojego Jedynego Syna, [a w dniu dzisiejszym] uwieńczyłeś ją najwyższą chwałą, q spraw, abyśmy zbawieni przez śmierć i zmartwychwstanie Jezusa Chrystusa, * przez Jej prośby mogli wejść do wiecznej chwały. Przez naszego Pana Jezusa Chrystusa, Twojego Syna, q który
z Tobą żyje i króluje w jedności Ducha Świętego, * Bóg, przez wszystkie
wieki wieków.



\startday[title={Droga przez pustynię}] % {{{1
Wezwanie: Uwielbiajmy Pana, * który kieruje narodami na ziemi
Hymn: piosenka dnia
Psalmodia
1 ant. Opowiedzieli nam nasi ojcowie o potędze Pana * i o cudach,
których On dokonał. / Alleluja.
Psalm 78, 1-39
Słuchaj, mój ludu, nauki mojej, *
nakłońcie wasze uszy na słowa ust moich;
Do przypowieści otworzę me usta, *
wyjawię tajemnice zamierzchłego czasu.
Tego, cośmy usłyszeli i poznali q
i co nam opowiedzieli nasi ojcowie, *
nie będziemy ukrywać przed ich synami.
Opowiemy przyszłemu pokoleniu q
chwałę Pana i Jego potęgę *
i cuda, których dokonał.
Albowiem nadał w Jakubie przykazania *
i ustanowił Prawo w Izraelu,
Aby to, co zlecił naszym ojcom, *
przekazali swym synom.
O tym ma wiedzieć przyszłe pokolenie, *
synowie, którzy się narodzą,
Że mają pokładać nadzieję w Bogu q
i nie zapominać dzieł Bożych, *
lecz strzec Jego poleceń.
A niech nie będą jak ich ojcowie *
pokoleniem opornych buntowników,
Pokoleniem, którego serce jest niestałe, *
a duch nie dochowuje wierności Bogu.
Synowie Efraima, uzbrojeni w łuki, *
zostali rozproszeni w dniu bitwy.
Nie zachowali przymierza z Bogiem *
i nie chcieli postępować według Jego Prawa.


Zapomnieli o Jego dziełach *
i o cudach, które im okazał.
Na oczach ich ojców uczynił cuda, *
w ziemi egipskiej, na polach Soanu.
Morze rozdzielił, by ich przeprowadzić, *
wody ustawił jak groble.
We dnie prowadził ich obłokiem, *
a przez całą noc blaskiem ognia.
Rozłupał skały w pustyni *
i jak wodną głębiną obficie ich napoił.
I wydobył ze skały strumienie, *
sprawił, że wody spłynęły jak rzeki.
Chwała Ojcu i Synowi, *
i Duchowi Świętemu.
Jak była na początku, teraz i zawsze, *
i na wieki wieków. Amen.
Ant. Opowiedzieli nam nasi ojcowie o potędze Pana / i o cudach,
których On dokonał. / Alleluja.
2 ant. Wysławiajcie Pana, Boga naszego, * pokłon Mu oddajcie
w Jego świątyni.
Psalm 99
Pan jest Królem, drżą narody, *
zasiada na tronie z cherubów, a ziemia się trzęsie.
Wielki jest Pan na Syjonie, *
wywyższony ponad wszystkie ludy.
Niech wielbią imię Twoje, wielkie i straszliwe, *
ono jest święte.
I króluje Potężny, który sprawiedliwość kocha: q
Tyś ład ustanowił, *
wymierzasz sprawiedliwość i prawo w Jakubie.
Wysławiajcie Pana, naszego Boga, q
padajcie przed podnóżkiem stóp Jego, *
bo On jest święty.
Wśród Jego kapłanów są Mojżesz i Aaron q
i Samuel wśród tych, którzy wzywali Jego imienia, *
wzywali Pana, a On ich wysłuchał.



Przemawiał do nich w słupie obłoku, *
a oni strzegli przykazań i prawa, które im nadał.
Boże, nasz Panie, Tyś ich wysłuchał, *
łaskę im okazałeś, lecz karałeś występki.
Wysławiajcie Pana, Boga naszego, q
pokłon oddajcie Jego świętej górze, *
bo Pan nasz i Bóg jest święty.
Chwała Ojcu i Synowi, *
i Duchowi Świętemu.
Jak była na początku, teraz i zawsze, *
i na wieki wieków. Amen.
Ant. Wysławiajcie Pana, Boga naszego, / pokłon Mu oddajcie
w Jego świątyni.
3 ant. Ich głos się rozchodzi po całej ziemi, * ich słowa aż po krańce
świata.
Psalm 19 A, 2-7
Niebiosa głoszą chwałę Boga, *
dzieło rąk Jego obwieszcza nieboskłon.
Dzień opowiada dniowi, *
noc nocy wiadomość przekazuje.
Nie są to słowa ani nie jest to mowa, *
których by dźwięku nie usłyszano.
Ich głos się rozchodzi po całej ziemi, *
ich słowa aż po krańce świata.
Tam słońcu namiot postawił, q
a ono jak oblubieniec wychodzi ze swej komnaty, *
cieszy się jak siłacz ruszający do biegu.
Ono wschodzi na krańcu nieba q
i biegnie aż po drugi kraniec, *
a nic przed jego żarem się nie schroni.
Chwała Ojcu i Synowi, *
i Duchowi Świętemu.
Jak była na początku, teraz i zawsze, *
i na wieki wieków. Amen.
Ant. Ich głos się rozchodzi po całej ziemi, * ich słowa aż po krańce
świata.



Czytanie: Mt 5, 14-16
Wy jesteście światłem świata. Nie może się ukryć miasto położone na górze. Nie zapala się też światła i nie stawia pod korcem, ale na świeczniku,
aby świeciło wszystkim, którzy są w domu. Tak niech świeci wasze światło
przed ludźmi, aby widzieli wasze dobre uczynki i chwalili Ojca waszego,
który jest w niebie.
Antyfona do pieśni Zachariasza:
Najdostojniejsza Królowo świata, * Maryjo, zawsze Dziewico, / Ty porodziłaś Chrystusa, naszego Pana i Zbawiciela.
Prośby
Oddając cześć naszemu Zbawicielowi, który narodził się z Maryi Dziewicy,
zanośmy do Niego pokorne błagania:
Niech Twoja Matka wstawia się za nami.
Jezu, Słońce sprawiedliwości, Twoje przyjście poprzedziła Niepokalana
Dziewica, jak pełna blasku jutrzenka,
– spraw, abyśmy zawsze żyli w promieniach Twojej światłości.
Dozwól nam, Panie, naśladować Twoją Matkę, która obrała najlepszą cząstkę,
– spraw, abyśmy za Jej przykładem szukali pokarmu dającego życie wieczne.
Zbawicielu świata, Ty mocą swojego odkupienia zachowałeś Twoją Matkę
od wszelkiej zmazy grzechu,
– zachowaj nas od skażenia grzechem.
Nasz Odkupicielu, Ty sprawiłeś, że Maryja Dziewica stała się godnym Ciebie mieszkaniem i przybytkiem Ducha Świętego,
– daj, abyśmy byli na wieki świątynią Twego Ducha.
Ojcze nasz.
Modlitwa:
Boże, Ty miłujesz ludzi, pokornie Cię błagamy, ześlij na nas obfitą łaskę
Ducha Świętego q i spraw, abyśmy postępując zgodnie z naszym powołaniem, dawali świadectwo Ewangelii * i dążyli z ufnością i w pokoju do zjednoczenia wszystkich wierzących. Przez naszego Pana Jezusa Chrystusa,
Twojego Syna, q który z Tobą żyje i króluje w jedności Ducha Świętego, *
Bóg, przez wszystkie wieki wieków.



Antyfony do pieśni Zachariasza
na niedziele zwykłe w okresie wakacyjnym
13 niedziela zwykła
Rok A: Kto was przyjmuje, * Mnie przyjmuje: / a kto Mnie przyjmuje,
przyjmuje Tego, który mnie posłał.
Rok B: Jezus widząc chorą kobietę * rzekł do niej: / Córko, twoja wiara
cię uzdrowiła, / idź w pokoju.
Rok C: Zostaw umarłym grzebanie ich umarłych, * a ty idź i głoś królestwo Boże.
14 niedziela zwykła
Rok A: Weźcie moje jarzmo na siebie * i uczcie się ode Mnie, / bo jestem cichy
i pokorny sercem.
Rok B: Zaprawdę powiadam wam: * Żaden prorok nie jest mile widziany
w swojej ojczyźnie.
Rok C: Gdy wejdziecie do jakiego domu, * najpierw mówcie: Pokój temu
domowi. / I wasz pokój spocznie na nim.
15 niedziela zwykła
Rok A: Jezus powiedział swoim uczniom: * Wam dano poznać tajemnice
królestwa niebieskiego, / innym zaś przez przypowieści.
Rok B: Uczniowie Jezusa wyszli * i wzywali do nawrócenia.
Rok C: Pewien Samarytanin * będąc w podróży przechodził obok
poranionego. / Gdy go zobaczył, wzruszył się głęboko / i opatrzył mu rany.
16 niedziela zwykła
Rok A: Królestwo niebieskie podobne jest do zaczynu, * który kobieta wzięła
i włożyła w trzy miary mąki, / aż się wszystko zakwasiło.
Rok B: Pójdźcie sami osobno na miejsce pustynne * i odpocznijcie nieco.
Rok C: Maria usiadła u nóg Pana * i przysłuchiwała się Jego słowom.
17 niedziela zwykła
Rok A: Królestwo niebieskie * jest podobne do sieci zarzuconej w morze, /
kiedy się napełniła, na brzeg ją wyciągnęli / i wybrali dobre ryby, a złe odrzucili.
Rok B: Pięcioma chlebami * i dwiema rybami / Chrystus nasycił pięć tysięcy ludzi.
Rok C: Proście, a będzie wam dane, * szukajcie, a znajdziecie, / kołaczcie,
a będzie wam otworzone.
18 niedziela zwykła
Rok A: Pięcioma chlebami * i dwiema rybami / Chrystus nasycił pięć tysięcy ludzi.
Rok B: Zaprawdę, zaprawdę powiadam wam: * Nie Mojżesz dał wam chleb
z nieba, / ale dopiero Ojciec mój da wam prawdziwy chleb z nieba.



Rok C: Gromadźcie sobie skarby w niebie, * gdzie ani mól, ani rdza
nie niszczą.
19 niedziela zwykła
Rok A: Jezus powiedział do zatrwożonych uczniów: * Odwagi, Ja jestem,
nie bójcie się.
Rok B: Zaprawdę, zaprawdę powiadam wam: * Kto we Mnie wierzy,
ma życie wieczne.
Rok C: Szczęśliwi słudzy, * których pan zastanie czuwających, gdy nadejdzie.
20 niedziela zwykła
Rok A: Kobieta kananejska przyszła do Jezusa * i oddała Mu pokłon,
mówiąc: / Panie, dopomóż mi.
Rok B: Ciało moje jest prawdziwym pokarmem, * a Krew moja jest prawdziwym
napojem. / Kto spożywa moje Ciało i pije Krew moją, ma życie wieczne.
Rok C: Chrzest mam przyjąć * i jakiej doznaję udręki, aż się to stanie.
21 niedziela zwykła
Rok A: Ty jesteś Piotr – Skała, * i na tej skale zbuduję mój Kościół.
Rok B: Nikt nie może przyjść do Mnie, * jeżeli mu to nie zostało dane
przez Ojca.
Rok C: Wielu przyjdzie ze wschodu i zachodu * i zasiądą z Abrahamem,
Izaakiem i Jakubem w królestwie niebieskim.
22 niedziela zwykła
Rok A: Cóż za korzyść odniesie człowiek, * choćby cały świat zyskał,
/ a na swej duszy szkodę poniósł?
Rok B: Przyjmijcie w duchu łagodności * zaszczepione w was słowo,
/ które ma moc zbawić dusze wasze.
Rok C: Każdy, kto się wywyższa, * będzie poniżony, / a kto się poniża
będzie wywyższony.

Antyfony do pieśni Zachariasza
na uroczystości w okresie wakacyjnym
29 czerwca – Świętych Apostołów Piotra i Pawła
Szymon Piotr powiedział do Jezusa: * Panie, do kogo pójdziemy? / Ty masz słowa
życia wiecznego. / A myśmy uwierzyli i poznali, że Ty jesteś Synem Bożym.
15 sierpnia – Wniebowzięcie Najświętszej Maryi Panny
Piękna i pełna blasku jesteś, Maryjo, * jak jutrzenka wznosisz się do nieba.



Wigilia modlitewna
przed Zesłaniem Ducha świętego
Hymn: O Stworzycielu Duchu przyjdź („Otwórzcie serca” nr 352)
Psalmodia
Ant. 1: Pan jest naszym Bogiem, * na wieki pamięta o swoim przymierzu
Psalm 105(104), 1-22
I
Sławcie Pana, wzywajcie jego imienia *
głoście jego dzieła wśród narodów
Śpiewajcie i grajcie mu psalmy *
rozsławiajcie wszystkie Jego cuda
Szczyćcie się Jego świętym imieniem *
Niech się weseli serce szukających Pana
Rozważcie o Panu i Jego potędze *
Zawsze szukajcie Jego oblicza
Pamiętajcie o cudach, które On uczynił *
o Jego znakach, o wyrokach ust Jego
Potomkowie Abrahama, słudzy Jego *
synowie Jakuba, Jego wybrańcy
On, Pan, jest naszym Bogiem *
Jego wyroki obejmują świat cały
Na wieki On pamięta o swoim przymierzu *
obietnicy danej tysiącu pokoleń
Przymierzu, które zawarł z Abrahamem *
przysiędze danej Izaakowi
Ustanowił je dla Jakuba jako prawo *
dla Izraela, jako wieczne przymierze
Gdy powiedział: „Dam tobie ziemię Kanaan *
na waszą własność dziedziczną”
Kiedy ich było niewielu *
nielicznych przybyszów w tym kraju
Wędrujących od plemienia do plemienia *
od królestwa do jeszcze innego ludu
Nikomu nie pozwolił ich uciskać *
karał z ich powodu królów



„Nie dotykajcie moich pomazańców *
i moim prorokom nie czyńcie krzywdy”
Potem głód przywołał na ziemię *
i odebrał im cały zapas chleba
Wysłał przed nimi męża: *
Józefa, którego sprzedano w niewolę
Kajdanami ścisnęli mu nogi *
jego kark zakuto w żelazo
Aż się spełniła jego przepowiednia *
i poświadczyło słowo Pana
Król posłał, aby go uwolnić *
wyzwolił go władca ludów
Ustanowił go panem nad swoim domem *
władcą całej swojej posiadłości
By według swej myśli pouczał dostojników *
a jego doradców uczył mądrości
Chwała Ojcu i Synowi *
i Duchowi Świętemu
Jak była na początku, teraz i zawsze *
i na wieki wieków. Amen
Ant. 1: Pan jest naszym Bogiem, na wieki pamięta o swoim przymierzu
Ant 2.: Pan Bóg pamięta * o swym świętym słowie
II
Potem Izrael wkroczył do Egiptu *
Jakub był gościem w kraju Chama
Bóg swój naród bardzo rozmnożył *
uczynił go mocniejszym od jego wrogów
Ich serce odmienił, aby znienawidzili lud Jego *
i wobec sług Jego postępowali zdradziecko
Posłał wtedy sługę swojego Mojżesza *
i Aarona wybranego przez siebie
Okazali wśród nich znaki *
i cuda w krainie Chama
Zesłał ciemności i mrok nastał *
lecz oni buntowali się przeciw Jego słowom
W krew zamienił ich wody *
i pozabijał ryby
Od żab zaroiła się ich ziemia *
nawet w komnatach królewskich


Rzekł i pojawiło się robactwo *
i w całym kraju ich komary
Zamiast deszczu grad zesłał *
palący ogień na ich ziemię
Powalił ich winnice i figi *
i drzewa połamał w ich kraju
Rzekł i nadciągnęła szarańcza *
nieprzeliczone roje świerszczy
Pożarły całą trawę w ich kraju *
i zjadły płody ich ziemi
Pobił wszystkich pierworodnych w ich kraju *
cały kwiat ich potęgi
A lud swój wyprowadził ze srebrem i złotem *
i nikt nie był słaby w jego pokoleniach
Egipcjanie cieszyli się z ich wyjścia *
bo ogarnął ich lęk przed nimi
Chmurę rozpostarł jako przykrycie *
i ogień, by świecił wśród nocy
Prosili i zesłał im przepiórki *
nasycił ich chlebem z nieba
Rozdarł skałę i trysnęła woda *
popłynęła pustynią jak rzeka
Pamiętał bowiem o swym świętym słowie *
danym Abrahamowi, swojemu słudze
I wyprowadził swój lud wśród radości *
z weselem swoich wybranych
Darował im ziemie narodów *
i zawładnęli dorobkiem ludów
By strzegli jego przykazań *
i zachowali prawa
Chwała Ojcu i Synowi *
i Duchowi Świętemu
Jak była na początku, teraz i zawsze *
i na wieki wieków. Amen
Ant. 2: Pan Bóg pamięta o swym świętym słowie
Ant. 3: Śpiewajcie Panu pieśń dziękczynną, * On gromadzi Izraela



Psalm 147(146-147)
Chwalcie Pana, bo dobrze jest śpiewać psalmy Bogu *
słodko jest Go wysławiać
Pan buduje Jeruzalem *
gromadzi rozproszonych z Izraela
On leczy złamanych na duchu *
i przewiązuje ich rany
On liczy wszystkie gwiazdy *
i każdej imię nadaje
Nasz Pan jest wielki i potężny *
a Jego mądrość niewypowiedziana
Pan dźwiga pokornych *
karki grzeszników zgina do ziemi
Śpiewajcie Panu pieśń dziękczynną *
Bogu naszemu grajcie na harfie
On niebo chmurami osłania *
przygotowuje deszcz na ziemi
Wzgórza trawą pokrywa *
i ziołami, które służą ludziom
On bydłu daje pokarm *
i pisklętom kruka, które wołają do Niego
Nie kocha się w sile rumaka *
ani w potędze męża
Upodobał sobie w tych, którzy cześć Mu oddają *
którzy ufają Jego dobroci
Chwal Jeruzalem Pana *
wysławiaj Twego Boga Syjonie
Umacnia bowiem zawory bram twoich *
i błogosławi synom twoim w tobie
Zapewnia pokój swoim granicom *
i wyborną pszenicą hojnie ciebie darzy
Zsyła na ziemię swoje polecenia *
a szybko mknie jego słowo
On prószy śniegiem jak wełną *
i szron rozsypuje jak popiół
On grad rozrzuca jak okruchy chleba *
od Jego mrozu ścinają się wody
Posyła słowo i lód topnieje *
powieje wiatrem i rzeki płyną



Oznajmił swoje słowo Jakubowi *
Izraelowi ustawy swe i wyroki
Nie uczynił tego dla innych narodów *
nie oznajmił im swoich wyroków
Chwała Ojcu i Synowi *
i Duchowi Świętemu
Jak była na początku, teraz i zawsze *
i na wieki wieków. Amen
Ant. 3: Śpiewajcie Panu pieśń dziękczynną, On gromadzi Izraela
Responsorium
K: Pan mówi: Oto ja jestem z wami
W: Przez wszystkie dni aż do skończenia świata

Czytanie z Księgi Wyjścia
Responsorium
L: Już was nie nazywam sługami
W: Ale przyjaciółmi, ponieważ poznaliście wszystko, co wam
uczyniłem. Przyjmijcie Ducha Świętego Pocieszyciela,
którego przyśle wam Ojciec, Alleluja
L: Jesteście przyjaciółmi moimi, gdy czynicie, co wam nakazałem
W: Przyjmijcie Ducha Świętego Pocieszyciela, którego przyśle
wam Ojciec, Alleluja

Czytanie z konstytucji dogmatycznej „Lumen Gentium”
Responsorium
L: Różne są działania
W: Lecz ten sam Bóg, sprawca wszystkiego we wszystkich.
Wszystkim zaś objawia się Duch dla wspólnego dobra, Alleluja
L: Wy jesteście Ciałem Chrystusa i poszczególnymi członkami
W: Wszystkim zaś objawia się Duch dla wspólnego dobra, Alleluja


Ewangelia
Hymn: „Ciebie Boga wysławiamy” („Otwórzcie serca” nr 113)
K: Błogosławmy Panu
W: Bogu niech będą dzięki


\stoptext
